\documentclass[
  aspectratio=1610,
]{beamer}

% chktex-file 1
% chktex-file 26

\usefonttheme{professionalfonts}
\usefonttheme[stillsansseriflarge, stillsansserifsmall]{serif}

%\usetheme{Frankfurt}
%\usecolortheme{dove}

\usepackage{amsmath, amssymb}
\usepackage{booktabs}
\usepackage{xcolor}

\usepackage[MnSymbol]{mathspec}
\setmathfont(Digits,Latin,Greek)[
  Numbers={Lining, Proportional}
]{Minion Pro}
\setmathrm{Minion Pro}

\defaultfontfeatures{
  Ligatures=TeX,
  Scale=MatchLowercase,
}
\setsansfont{Gill Sans}
\setmainfont{Minion Pro}
\setmonofont{Fira Mono}

\usepackage{xltxtra}
\usepackage{polyglossia}
\setdefaultlanguage{german}

\usepackage{hyperref}

\setbeamertemplate{navigation symbols}{}

\graphicspath{{./bilder/}}

% Schachkram:
% siehe: http://mirrors.ctan.org/macros/latex/contrib/chessboard/chessboard.pdf
\usepackage{chessboard}
\usepackage{xskak}

\newcommand{\wplogo}{\includegraphics[height=2ex]{enwiki.png}}

\author{Martin Darmüntzel}
\title{Schach für Anfänger}
\subtitle{Einführung und Regeln}
%\institute{Universität Rostock}
%\date{6. September 2019}

\begin{document}

\begin{frame}
  \titlepage{}
\end{frame}

\begin{frame}{Inhalt des Kurses}
  \begin{itemize}
    \item Regeln
    \item Endspiele
    \item Taktiken
    \item Puzzle lösen
    \item Eröffnungen
    \item Schachvarianten
  \end{itemize}
\end{frame}

\begin{frame}[<+->]{Regeln – Grundspielregeln}
  \framesubtitle{Artikel 1: Wesen und Ziele des Schachspiels}
  \begin{itemize}
    \item[1.1] Das Schachspiel wird zwischen zwei Gegnern gespielt, die ihre Figuren auf
      einem quadratischen Spielbrett, \emph{Schachbrett} genannt, ziehen.
    \item[1.2] Der Spieler mit den hellen Figuren (Weiß) führt den ersten Zug aus, dann
      ziehen die Spieler abwechselnd, wobei der Spieler mit den dunklen Figuren (Schwarz)
      den nächsten Zug ausführt.
    \item[1.3] Ein Spieler ist am \emph{am Zug}, sobald der Zug seines Gegners ausgeführt
      worden ist.
  \end{itemize}
\end{frame}

\begin{frame}[<+->]{Regeln – Grundspielregeln}
  \framesubtitle{Artikel 1: Wesen und Ziele des Schachspiels}
  \begin{itemize}
    \item[1.4] Das Ziel eines jeden Spielers ist es, den gegnerischen König so
      \emph{anzugreifen}, dass der Gegner keinen regelmäßigen Zug zur Verfügung hat.
    \item[1.4.1] Der Spieler, der dieses Ziel erreicht, hat den gegnerischen König
      \emph{matt gesetzt} und das Spiel gewonnen. Es ist nicht erlaubt, den eigenen
      König im Angriff stehen zu lassen, den eigenen König einem Angriff auszusetzen
      oder den König des Gegners zu schlagen.
    \item[1.4.2] Der Gegner, dessen König matt gesetzt worden ist, hat das Spiel
      verloren.
  \end{itemize}
\end{frame}

\begin{frame}[<+->]{Regeln – Grundspielregeln}
  \framesubtitle{Artikel 1: Wesen und Ziele des Schachspiels}
  \begin{itemize}
    \item[1.5] Ist eine Stellung erreicht, in der keinem der beiden Spieler das Mattsetzen
      des gegnerischen Königs mehr möglich ist, ist das Spiel \emph{remis}
      (unentschieden siehe Artikel 5.2.2).
  \end{itemize}
\end{frame}

\begin{frame}[<+->]{Regeln – Grundspielregeln}
  \framesubtitle{Artikel 2: Die Anfangsstellung der Figuren auf dem Schachbrett}
  \begin{itemize}
    \item[2.1] Das Schachbrett besteht aus einem $8 \times 8$ Gitter von $64$ gleich großen
      Quadraten, die abwechselnd hell und dunkel sind (die \emph{weißen} und die
      \emph{schwarzen} Felder).\medskip

      Das Schachbrett wird so zwischen die beiden Spieler gelegt, dass auf der Seite vor
      einem Spieler das rechte Eckfeld weiß ist.
  \end{itemize}
\end{frame}

\begin{frame}[<+->]{Regeln – Grundspielregeln}
  \framesubtitle{Artikel 2: Die Anfangsstellung der Figuren auf dem Schachbrett}
  \begin{itemize}
    \item[2.2] Zu Beginn der Partie hat der eine Spieler 16 helle (\emph{weiße}), der
      andere 16 dunkle (\emph{schwarze}) Figuren.

      Diese Figuren sind die folgenden:
      \begin{center}
        \begin{tabular}{llllllll}
          ein weißer König    & \WhiteKingOnWhite   & ein schwarzer König    & \BlackKingOnWhite    & K & K\\
          eine weiße Dame     & \WhiteQueenOnWhite  & eine schwarze Dame     & \BlackQueenOnWhite   & D & Q\\
          zwei weiße Türme    & \WhiteRookOnWhite   & zwei schwarze Türme    & \BlackRookOnWhite    & T & R\\
          zwei weiße Läufer   & \WhiteBishopOnWhite & zwei schwarze Läufer   & \BlackBishopOnWhite  & L & B\\
          zwei weiße Springer & \WhiteKnightOnWhite & zwei schwarze Springer & \BlackKnightOnWhite  & S & N\\
          acht weiße Bauern   & \WhitePawnOnWhite   & acht schwarze Bauern   & \BlackPawnOnWhite           \\
        \end{tabular}
      \end{center}
  \end{itemize}
\end{frame}

\begin{frame}[<+->]{Regeln – Grundspielregeln}
  \framesubtitle{Artikel 2: Die Anfangsstellung der Figuren auf dem Schachbrett}
  \begin{itemize}
    \item[2.3] Die Anfangsstellung der Figuren auf dem Schachbrett ist die folgende:
      \begin{center}
        \newchessgame[]
        \chessboard[showmover=false]
      \end{center}
  \end{itemize}
\end{frame}

\begin{frame}[<+->]{Regeln – Grundspielregeln}
  \framesubtitle{Artikel 2: Die Anfangsstellung der Figuren auf dem Schachbrett}
  \begin{itemize}
    \item[2.4] Die acht senkrechten Spalten von Feldern heißen \emph{Linien}; die acht
      waagerechten Zeilen von Feldern heißen \emph{Reihen}. Eine geradlinige Folge von
      Felder gleicher Farbe, von einem Rand des Schachbrettes zum benachbarten Rand
      verlaufend, heißt \emph{Diagonale}.
  \end{itemize}
\end{frame}

\begin{frame}[<+->]{Regeln – Grundspielregeln}
  \framesubtitle{Artikel 3: Die Gangart der Figuren}
  \begin{itemize}
    \item[3.1] Es ist nicht gestattet eine Figur auf ein Feld zu ziehen, das bereits von
      einer Figur der gleichen Farbe besetzt ist.

    \item[3.1.1] Wenn eine Figur auf ein Feld zieht, das von einer gegnerischen Figur
      besetzt ist, wird letztere geschlagen und als Teil desselben Zuges vom Schachbrett
      entfernt.
    \item[3.1.2] Eine Figur greift eine gegnerische Figur an, wenn sie auf jenem Feld
      gemäß Artikel 3.2 bis 3.8 schlagen könnte.
    \item[3.1.3] Eine Figur greift ein Feld an, auch wenn sie am Zug gehindert ist, weil
      sie andernfalls den eigenen König im Angriff stehen lassen oder ihn einem Angriff
      aussetzen würde.
  \end{itemize}
\end{frame}

\begin{frame}[<+->]{Regeln – Grundspielregeln}
  \framesubtitle{Artikel 3: Die Gangart der Figuren}
  \begin{itemize}
    \item[3.2]
      Der Läufer darf auf ein beliebiges anderes Feld entlang einer der Diagonalen ziehen,
      auf der er steht.
      \begin{center}
        \chessboard[
          setpieces={be4},
          showmover=false,
          padding=-0.8ex,
          pgfstyle={[fill]circle},
          markfields={d5, c6, b7, a8, f3, g2, h1, b1, c2, d3, f5, g6, h7}
        ]
      \end{center}
  \end{itemize}
\end{frame}

\begin{frame}[<+->]{Regeln – Grundspielregeln}
  \framesubtitle{Artikel 3: Die Gangart der Figuren}
  \begin{center}
    \chessboard[
      setfen=k1K5/8/4B3/2B5/8/8/8/8 w - - 0 1, % chktex 8
      moveid=1w,
    ]

    Bester Zug für Weiß?
  \end{center}
\end{frame}

\begin{frame}[<+->]{Regeln – Grundspielregeln}
  \framesubtitle{Artikel 3: Die Gangart der Figuren}
  \begin{itemize}
    \item[3.3] Der Turm darf auf ein beliebiges Feld entlang der Linie oder der Reihe
      ziehen, auf der er steht.
      \begin{center}
        \chessboard[
          setpieces={rd3},
          showmover=false,
          padding=-0.8ex,
          pgfstyle={[fill]circle},
          markfields={d1, d2, d4, d5, d6, d7, d8, a3, b3, c3, e3, f3, g3, h3}
        ]
      \end{center}
  \end{itemize}
\end{frame}

\begin{frame}[<+->]{Regeln – Grundspielregeln}
  \framesubtitle{Artikel 3: Die Gangart der Figuren}
  \begin{center}
    \chessboard[
      setfen=k/8/1R/2R/8/8/8/7K w - - 0 1, % chktex 8
      moveid=1w,
    ]

    Bester Zug für Weiß?
  \end{center}
\end{frame}

\begin{frame}[<+->]{Regeln – Grundspielregeln}
  \framesubtitle{Artikel 3: Die Gangart der Figuren}
  \begin{itemize}
    \item[3.4] Die Dame darf auf ein beliebiges anderes Feld entlang der Linie, der Reihe
      oder einer der Diagonalen ziehen, auf der sie steht.
      \begin{center}
        \chessboard[
          setpieces={qe4},
          showmover=false,
          padding=-0.8ex,
          pgfstyle={[fill]circle},
          markfields={
            a4, b4, c4, d4, f4, g4, h4,
            e1, e2, e3, e5, e6, e7, e8,
            a8, b7, c6, d5, f3, g2, h1,
            b1, c2, d3, f5, g6, h7
          }
        ]
      \end{center}
  \end{itemize}
\end{frame}

\begin{frame}[<+->]{Regeln – Grundspielregeln}
  \framesubtitle{Artikel 3: Die Gangart der Figuren}
  \begin{center}
    \chessboard[
      setfen=k/2K/8/8/1Q/8/8/8 w - - 0 1, % chktex 8
      moveid=1w,
    ]

    Auf wie viele Arten kann Weiß matt setzen?
  \end{center}
\end{frame}

\begin{frame}[<+->]{Regeln – Grundspielregeln}
  \framesubtitle{Artikel 3: Die Gangart der Figuren}
  \begin{itemize}
    \item[3.5] Beim Ausführen dieser Züge dürfen Dame, Turm und Läufer nicht über
      dazwischen stehende Figuren hinweg ziehen.
  \end{itemize}
\end{frame}

\begin{frame}{Regeln – Grundspielregeln}
  \framesubtitle{Artikel 3: Die Gangart der Figuren}
  \begin{itemize}
    \item[3.6] Der Springer darf auf eines der Felder ziehen, die seinem Standfeld am
      nächsten, aber nicht auf gleicher Linie, Reihe oder Diagonalen mit diesem liegen.
      \begin{columns}
        \begin{column}{0.6\textwidth}
          \begin{center}
            \chessboard[
              setpieces={Nc3, bf8, ng8, rh8, pf7, pg7, ph7},
              showmover=false,
              padding=-0.8ex,
              markstyle=circle,
              markfields={a2, b1, a4, b5, d5, e4, e2, d1},
              pgfstyle={[fill]circle},
              markfields={e7, f6, h6},
            ]
          \end{center}
        \end{column}

        \pause

        \begin{column}{0.4\textwidth}
          \begin{center}
            \chessboard[
              smallboard,
              setpieces={Nd4},
              showmover=false,
              padding=-0.8ex,
              markstyle=circle,
              markfields={b3, b5, c2, c6, e2, e6, f3, f5},
              addpgf={
                \tikz[overlay] \draw [blue, line width=0.1em] (d4) circle (2.236em);
              },
            ]

            Der Springer springt im Kreis.
          \end{center}
        \end{column}
      \end{columns}
  \end{itemize}
\end{frame}

\begin{frame}[<+->]{Regeln – Grundspielregeln}
  \framesubtitle{Artikel 3: Die Gangart der Figuren}
  \begin{center}
    \chessboard[
      setfen=5rkb/5ppp/8/3N/8/5n/r/7K w - - 0 1, % chktex 8
      moveid=1w,
    ]

    Bester Zug für Weiß?
  \end{center}
\end{frame}

\begin{frame}{Regeln – Grundspielregeln}
  \framesubtitle{Artikel 3: Die Gangart der Figuren}
  \begin{columns}
    \begin{column}{0.6\textwidth}
      \begin{itemize}[<+->]
        \item[3.7.1] Der Bauer darf vorwärts auf das Feld direkt vor ihm auf derselben
          Linie ziehen, sofern dieses Feld nicht besetzt ist, oder
        \item[3.7.2] in seinem ersten Zug entweder wie unter 3.7.1 ziehen oder um zwei
          Felder auf derselben Linie vorrücken, sofern beide Felder nicht besetzt sind,
          oder
        \item[3.7.3] auf ein von einer gegnerischen Figur besetztes Feld diagonal vor ihm
          auf einer benachbarten Linie ziehen, indem er die Figur schlägt.
      \end{itemize}
    \end{column}

    \begin{column}{0.4\textwidth}
      \begin{center}
        \chessboard[
          smallboard,
          setpieces={Pc2, pg5},
          showmover=false,
          padding=-0.8ex,
          markstyle=circle,
          markfields={c3, c4},
          pgfstyle={[fill]circle},
          markfields={g4},
          markstyle=cross,
          shorten=0.6ex,
          markfields={b3, d3, f4, h4},
        ]
      \end{center}
    \end{column}
  \end{columns}
\end{frame}

\begin{frame}{Regeln – Grundspielregeln}
  \framesubtitle{Artikel 3: Die Gangart der Figuren}
  \begin{columns}
    \begin{column}{0.6\textwidth}
      \begin{itemize}
        \item[3.7.4.1] Ein Bauer, der auf derselben Reihe auf einem unmittelbar
          angrenzenden Feld wie ein gegnerischer Bauer, der soeben zwei Felder von seiner
          Anfangsstellung vorgerückt ist, steht, darf diesen gegnerischen Bauern so
          schlagen, als ob letzterer nur um ein Feld vorgerückt wäre.

        \item[3.7.4.2] Dieses Schlagen ist nur in dem unmittelbar nachfolgenden Zug
          regelmäßig und wird \emph{Schlagen en passant} genannt.
      \end{itemize}
    \end{column}

    \pause

    \begin{column}{0.4\textwidth}
      \begin{center}
        \chessboard[
          smallboard,
          setpieces={Pd5, pe7},
          showmover=false,
          padding=-0.8ex,
          pgfstyle={[fill]circle},
          markfields={e5},
          markstyle=cross,
          shorten=0.6ex,
          markfields={e6},
        ]
      \end{center}
    \end{column}
  \end{columns}
\end{frame}

\begin{frame}[<+->]{Regeln – Grundspielregeln}
  \framesubtitle{Artikel 3: Die Gangart der Figuren}
  \begin{center}
    \chessboard[
      setfen=8/4p/3P/5P/8/8/8/8 w - - 0 1, % chktex 8
      moveid=1w,
    ]

    Schwarz spielt e7–e5. Welche möglichen Züge hat Weiß?
  \end{center}
\end{frame}

\begin{frame}[<+->]{Regeln – Grundspielregeln}
  \framesubtitle{Artikel 3: Die Gangart der Figuren}
  \begin{itemize}
    \item[3.7.5.1] Wenn ein Spieler, der am Zug ist, seinen Bauern auf die von seiner
      Anfangsstellung entfernteste Reihe zieht, muss er diesen als Teiler desselben Zuges
      gegen eine Dame, einen Turm, Läufer oder Springer derselben Farbe auf dem
      Ankunftsfeld austauschen. Dieses wird Umwandlungsfeld genannt.
    \item[3.7.5.2] Die Auswahl des Spielers ist nicht auf bereits geschlagene Figuren
      beschränkt.
    \item[3.7.5.3] Dieser Austausch eines Bauern für eine andere Figur wird
      \emph{Umwandlung} genannt. Die Wirkung der neuen Figur tritt sofort ein.
  \end{itemize}
\end{frame}

\begin{frame}[<+->]{Regeln – Grundspielregeln}
  \framesubtitle{Artikel 3: Die Gangart der Figuren}
  \begin{center}
    \chessboard[
      setfen=7k/4P1pp/8/8/8/5n/2r/7K w - - 0 1, % chktex 8
      moveid=1w,
    ]

    Bester Zug für Weiß?
  \end{center}
\end{frame}

\begin{frame}[<+->]{Regeln – Grundspielregeln}
  \framesubtitle{Artikel 3: Die Gangart der Figuren}
  \begin{itemize}
    \item[3.8] Es gibt zwei verschiedene Arten den König zu ziehen:
    \item[3.8.1] durch Ziehen auf ein beliebiges angrenzendes Feld,

      \begin{center}
        \chessboard[
          setpieces={Kc3, ke8},
          showmover=false,
          padding=-0.8ex,
          markstyle=circle,
          markfields={b2, b3, b4, c2, c4, d2, d3, d4},
          pgfstyle={[fill]circle},
          markfields={d7, d8, e7, f7, f8},
        ]
      \end{center}
  \end{itemize}
\end{frame}

\begin{frame}[<+->]{Regeln – Grundspielregeln}
  \framesubtitle{Artikel 3: Die Gangart der Figuren}
  \begin{itemize}
    \item[3.8.2] durch Rochieren.
      Die Rochade ist ein Zug des Königs und eines gleichfarbigen Turmes auf der
      ersten Reihe des Spielers, der als ein Königszug gilt und folgendermaßen
      ausgeführt wird: Der König wird von seiner Anfangsstellung um zwei Felder in
      Richtung des Turmes, der auf seiner Anfangsstellung stehen muss, hin versetzt;
      dann wird dieser Turm auf das Feld gesetzt, das der König soeben überquert hat.
  \end{itemize}
\end{frame}

\begin{frame}[<+->]{Regeln – Grundspielregeln}
  \framesubtitle{Artikel 3: Die Gangart der Figuren}
  \begin{center}
    \begin{tabular}{cc}
      Vor schwarzer langer Rochade & Nach schwarzer langer Rochade\\
      \chessboard[
        fontsize=18pt,
        labelleft=false,
        labelbottom=false,
        margin=false,
        setpieces={Ke1, Rh1, ke8, ra8},
        showmover=false,
      ]

      &

      \chessboard[
        fontsize=18pt,
        labelleft=false,
        labelbottom=false,
        margin=false,
        setpieces={Kg1, Rf1, kc8, rd8},
        showmover=false,
      ]
      \\
      Vor weißer kurzer Rochade & Nach weißer kurzer Rochade
    \end{tabular}
  \end{center}
\end{frame}

\begin{frame}[<+->]{Regeln – Grundspielregeln}
  \framesubtitle{Artikel 3: Die Gangart der Figuren}
  \begin{center}
    \begin{tabular}{cc}
      Vor schwarzer kurzer Rochade & Nach schwarzer kurzer Rochade\\
      \chessboard[
        fontsize=18pt,
        labelleft=false,
        labelbottom=false,
        margin=false,
        setpieces={Ke1, Ra1, ke8, rh8},
        showmover=false,
      ]

      &

      \chessboard[
        fontsize=18pt,
        labelleft=false,
        labelbottom=false,
        margin=false,
        setpieces={Kc1, Rd1, kg8, rf8},
        showmover=false,
      ]
      \\
      Vor weißer langer Rochade & Nach weißer langer Rochade
    \end{tabular}
  \end{center}
\end{frame}

\begin{frame}[<+->]{Regeln – Grundspielregeln}
  \framesubtitle{Artikel 3: Die Gangart der Figuren}
  \begin{itemize}
    \item[3.8.2.1] Das Recht zu rochieren ist verloren:
    \item[3.8.2.1.1] wenn der König bereits gezogen hat, oder
    \item[3.8.2.1.2] mit einem Turm, der bereits gezogen hat.
  \end{itemize}
\end{frame}

\begin{frame}{Regeln – Grundspielregeln}
  \framesubtitle{Artikel 3: Die Gangart der Figuren}
  \begin{itemize}
    \item[3.8.2.2] Die Rochade ist vorübergehend verhindert,
    \item[3.8.2.2.1] wenn das Standfeld des Königs oder das Feld, das er überqueren muss,
      oder sein Zielfeld von einer oder mehreren gegnerischen Figuren angegriffen wird,
    \item[3.8.2.2.1] wenn sich zwischen dem König und dem Turm, mit dem rochiert werden
      soll, irgendeine Figur befindet.
  \end{itemize}
\end{frame}

\begin{frame}[<+->]{Regeln – Grundspielregeln}
  \framesubtitle{Artikel 3: Die Gangart der Figuren}
  \begin{center}
    \chessboard[
      setfen=r3k2r/ppp3pp/3pN3/3n1P2/6q1/Q7/PPP3PP/R3K2R w KQkq - 0 1, % chktex 8
      moveid=1w,
    ]

    Wer darf wie rochieren?
  \end{center}
\end{frame}

\begin{frame}[<+->]{Regeln – Grundspielregeln}
  \framesubtitle{Artikel 3: Die Gangart der Figuren}
  \begin{itemize}
    \item[3.9.1] Ein König steht \emph{im Schach}, wenn er von einer oder mehreren
      gegnerischen Figuren angegriffen wird, auch wenn diese selbst nicht auf das vom
      König besetzte Feld ziehen können, weil sie anderenfalls den eigenen König im
      Angriff stehen lassen oder diesen einem Angriff aussetzen würden.
    \item[3.9.2] Keine Figur darf einen Zug machen, der entweder den König derselben Farbe
      einem Schachgebot aussetzt oder diesen in einem Schachgebot stehen lässt.
  \end{itemize}
\end{frame}

\begin{frame}[<+->]{Regeln – Grundspielregeln}
  \framesubtitle{Artikel 3: Die Gangart der Figuren}
  \begin{itemize}
    \item[3.10.1] Ein Zug ist regelmäßig, wenn die maßgeblichen Bedingungen der Artikel
      3.1 bis 3.9 erfüllt wurden.
    \item[3.10.2] Ein Zug ist regelwidrig, wenn er eine der maßgeblichen Bedingungen der
      Artikel 3.1 bis 3.9 nicht erfüllt.
    \item[3.10.3] Eine Stellung ist regelwidrig, wenn sie nicht durch irgendeine Folge
      regelgemäßer Züge erreicht werden kann.
  \end{itemize}
\end{frame}

\begin{frame}[<+->]{Übungsaufgaben}
  \begin{center}
    \chessboard[
      setfen=6nr/5Ppk/2r4p/7P/8/8/PP/K w - - 0 1, % chktex 8
      moveid=1w,
    ]

    Bester Zug für Weiß?
  \end{center}
\end{frame}

\begin{frame}[<+->]{Übungsaufgaben}
  \begin{center}
    \chessboard[
      setfen=3k/8/8/8/8/8/1r/R3K w - - 0 1, % chktex 8
      moveid=1w,
    ]

    Bester Zug für Weiß?
  \end{center}
\end{frame}

\begin{frame}[<+->]{Übungsaufgaben}
  \begin{center}
    \chessboard[
      setfen=8/ppp/1kn/8/KPP/8/8/8 w - - 0 1, % chktex 8
      moveid=1w,
    ]

    Bester Zug für Weiß?
  \end{center}
\end{frame}

\begin{frame}[<+->]{Übungsaufgaben}
  \begin{center}
    \chessboard[
      setfen=8/3P1kp/2q2p/8/8/8/5PPP/6K w - - 0 1, % chktex 8
      moveid=1w,
    ]

    Bester Zug für Weiß?
  \end{center}
\end{frame}

\begin{frame}[<+->]{Übungsaufgaben}
  \begin{center}
    \chessboard[
      setfen=1nb/k1pq/7p/PP/7K/2P2QP/3PB/8 w - - 0 1, % chktex 8
      moveid=1w,
    ]

    Schwarz spielt c7–c5. Was ist der beste Zug für Weiß?
  \end{center}
\end{frame}

\begin{frame}[<+->]{Regeln – weitere ausgewählte Regeln}
  \framesubtitle{Berührt – geführt}
  \begin{itemize}
    \item[4.1] Jeder Zug muss mit einer Hand alleine ausgeführt werden.
    \item[4.2.1] Nur der Spieler, der am Zug ist, darf eine oder mehrere Figuren auf ihren
      Felder zurechtrücken, vorausgesetzt, dass er seine Absicht im Voraus bekannt gibt
      (zum Beispiel durch die Ankündigung \emph{j’adoube} oder \emph{ich korrigiere}).
    \item[4.2.2] Jede andere Berührung einer Figur gilt als absichtliche Berührung, außer
      dies geschieht offensichtlich aus Versehen.
  \end{itemize}
\end{frame}

\begin{frame}[<+->]{Regeln – weitere ausgewählte Regeln}
  \framesubtitle{Ende der Partie}
  \begin{itemize}
    \item[5.1.1] Die Partie ist von dem Spieler gewonnen, der den gegnerischen König
      mattgesetzt hat.
      Damit ist die Partie sofort beendet,
      \textcolor{lightgray}{
        vorausgesetzt, dass der Zug, der die Mattstellung herbeigeführt hat, mit Artikel 3
        und den Artikeln 4.2 bis 4.7 übereinstimmte.
    }
    \item[5.1.2] Die Partie ist von dem Spieler gewonnen, dessen Gegner erklärt, dass er
      aufgebe. Damit ist die Partie sofort beendet.
  \end{itemize}
\end{frame}

\begin{frame}[<+->]{Regeln – weitere ausgewählte Regeln}
  \framesubtitle{Ende der Partie}
  \begin{itemize}
    \item[5.2.1] Die Partie ist remis, wenn der Spieler, der am Zug ist, keinen
      regelgemäßen Zug zur Verfügung hat und sein König nicht im Schach steht. Eine solche
      Stellung heißt \emph{Pattstellung}.
      Damit ist die Partie sofort beendet,
        \textcolor{lightgray}{
        vorausgesetzt, dass der Zug, der die Mattstellung herbeigeführt hat, mit Artikel 3
        und den Artikeln 4.2 bis 4.7 übereinstimmte.
      }
  \end{itemize}
\end{frame}

\begin{frame}[<+->]{Regeln – Grundspielregeln}
  \framesubtitle{Artikel 3: Die Gangart der Figuren}
  \begin{center}
    \chessboard[
      setfen=k/2K/8/8/1Q/8/8/8 w - - 0 1, % chktex 8
      moveid=1w,
    ]

    Welcher Zug von Weiß führt zu einer Pattstellung?
  \end{center}
\end{frame}

\begin{frame}[<+->]{Regeln – weitere ausgewählte Regeln}
  \framesubtitle{Ende der Partie}
  \begin{itemize}
    \item[5.2.2] Die Partie ist remis, sobald eine Stellung entstanden ist, in welcher
      keiner der Spieler den gegnerischen König mit irgendeiner Folge regelmäßiger Züge
      matt setzen kann. Eine solche Stellung heißt \emph{tote Stellung}.
      Damit ist die Partie sofort beendet,
      \textcolor{lightgray}{
        vorausgesetzt, dass der Zug, der die Mattstellung herbeigeführt hat, mit Artikel 3
        und den Artikeln 4.2 bis 4.7 übereinstimmte.
      }
    \item[5.2.3] Die Partie ist remis durch eine von den beiden Spielern während der
      Partie getroffene Übereinkunft, sofern beide Spieler mindestens einen Zug ausgeführt
      haben. Damit ist die Partie sofort beendet.
  \end{itemize}
\end{frame}

\begin{frame}[<+->]{Regeln – weitere ausgewählte Regeln}
  \framesubtitle{Ende der Partie}
  \begin{itemize}
    \item[9.2.1] Die Partie ist remis aufgrund eines korrekten Antrages des Spielers, der
      am Zug ist, wenn die gleiche Stellung mindestens zum dritten Mal (nicht
      notwendigerweise durch Zugwiederholung)
    \item[9.2.1.1] sogleich entstehen wird, falls der Spieler als erstes seinen Zug, der
      nicht geändert werden kann, auf sein Partieformular schreibt und dem Schiedsrichter
      seine Absicht erklärt, diesen Zug ausführen zu wollen, oder
    \item[9.2.1.2] soeben entstanden ist und der Antragsteller am Zug ist.
  \end{itemize}

  \pause

  \begin{block}{Übersetzung}
    Wenn sich die Stellung zum dritten Mal wiederholt, darf ein Spieler remis
    beantragen und die Partie endet remis.
  \end{block}

  \pause

  Wann sind zwei Stellungen gleich?
\end{frame}

\begin{frame}[<+->]{Regeln – weitere ausgewählte Regeln}
  \framesubtitle{Ende der Partie}
  \begin{center}
    \chessboard[
      setfen=6k/6p/6Q/8/3q/8/rr5P/7K w - - 0 1, % chktex 8
      moveid=1w,
    ]

    Wie kann Weiß noch ein remis holen?
  \end{center}
\end{frame}

\begin{frame}[<+->]{Regeln – weitere ausgewählte Regeln}
  \framesubtitle{Ende der Partie}
  \begin{itemize}
    \item[9.3] Die Partie ist remis aufgrund eines korrekten Antrages des Spieler, der am
      Zug ist, falls
    \item[9.3.1] er einen Zug, der nicht geändert werden kann, auf sein Partieformular
      schreibt und dem Schiedsrichter seine Absicht erklärt, diesen Zug ausführen zu wollen,
      der zur Folge habe, dass dann die letzten 50 aufeinanderfolgenden Züge eines jeden
      Spieler ausgeführt worden sind, ohne dass ein Bauer gezogen hat und ohne dass eine
      Figur geschlagen worden ist, oder
    \item[9.3.2] die letzten 50 aufeinanderfolgenden Züge von jedem Spieler abgeschlossen
      worden sind, ohne dass ein Bauer gezogen hat und ohne dass eine Figure geschlagen
      worden ist.
  \end{itemize}

  \begin{block}{Übersetzung}
    Wenn in den letzten 50 Zügen weder eine Figur geschlagen noch ein Bauer bewegt wurde,
    darf ein Spieler remis beantragen und die Partie endet remis.
  \end{block}
\end{frame}

\begin{frame}[<+->]{Regeln – weitere ausgewählte Regeln}
  \framesubtitle{Ende der Partie}
  \begin{itemize}
    \item[9.6] Falls eine oder beide der folgenden Situationen auftreten, ist die Partie
      remis:
    \item[9.6.1] sobald eine gleiche Stellung, entsprechend Artikel 9.2.2, mindestens
      fünfmal entstanden ist,
    \item[9.6.2] sobald wenigstens 75 Züge von jedem Spieler ausgeführt worden sind, ohne
      dass ein Bauer gezogen hat und ohne dass eine Figur geschlagen worden ist.
      Wenn der letzte Zug matt setzt, hat dies Vorrang.
  \end{itemize}
\end{frame}

\begin{frame}[<+->]{Regeln – weitere ausgewählte Regeln}
  \framesubtitle{Etiquette}
  \begin{itemize}
    \item[11.1] Die Spieler dürfen nichts unternehmen, das dem Ansehen des Schachspiels
      abträglich sein könnte.
    \item[11.5] Es ist verboten, den Gegner auf irgendwelche Art abzulenken oder zu
      stören.  Dazu gehört auch ungerechtfertigtes Antragstellen oder ungerechtfertigtes
      Anbieten von Remis oder das Mitbringen einer Geräuschquelle in den Spielbereich.
    \item[11.7] Andauernde Weigerung eines Spieler, sich an die Schachregeln zu halten,
      wird mit Partieverlust bestraft. Die vom Gegner erzielte Punktzahl wird vom
      Schiedsrichter bestimmt.
    \item[11.8] Wenn sich beide Spieler gemäß Artikel 11.7 schuldig machen, wird für beide
      das Spiel für verloren erklärt.
  \end{itemize}
\end{frame}

\end{document}
