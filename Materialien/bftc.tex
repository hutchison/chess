\documentclass[
%% alle weiteren Papierformat einstellbar:
a4paper, %apaper,
%% keine Seitenzahlen:
%empty,
%% Schriftgröße (12pt, 11pt (Standard)):
11pt,
%% kleinere Überschriften:
%smallheadings,
]
{scrartcl}

\input{packages.latex}

\setchessboard{%
  showmover=false
}

\title{Schachnotizen}
\author{Martin Darmüntzel}

\begin{document}

\maketitle

\tableofcontents

\section{Bobby Fischer Teaches Chess}

Die folgenden Probleme sind aus
\href{https://en.wikipedia.org/wiki/Bobby_Fischer_Teaches_Chess}{Bobby Fischer
Teaches Chess}.

\pagebreak

Schau dir die folgende Position an.
Entscheide, ob der schwarze König dem Schach entkommen kann.

\begin{center}
  \chessboard[setfen=7k/6pQ/8/5B2/8/8/8/4K3]
\end{center}

Die weiße Dame gibt Schach. Der schwarze König:
\begin{itemize}
  \item[$\square$] kann schlagen
  \item[$\square$] kann nicht schlagen
\end{itemize}

\pagebreak

kann nicht schlagen

Die Dame ist durch den Läufer gedeckt, daher kann der König nicht schlagen.
Tatsächlich kann der schwarze König dem Schach auch nicht entkommen und ist
somit schachmatt.

\pagebreak

Schau dir die folgende Position an.
Entscheide, ob der schwarze König dem Schach entkommen kann.

\begin{center}
  \chessboard[setfen=k/Q/8/8/2B/8/8/4K]
\end{center}

Die weiße Dame gibt Schach. Der schwarze König:
\begin{itemize}
  \item[$\square$] kann schlagen
  \item[$\square$] kann nicht schlagen
\end{itemize}

\pagebreak

kann schlagen

Die Dame ist nicht durch den Läufer gedeckt.

\pagebreak

Schau dir die folgende Position an.
Entscheide, ob der schwarze König dem Schach entkommen kann.

\begin{center}
  \chessboard[setfen=6Rk/7p/8/5N/8/8/8/6K]
\end{center}

Der schwarze König:
\begin{itemize}
  \item[$\square$] kann den weißen Turm schlagen
  \item[$\square$] kann den weißen Turm nicht schlagen
\end{itemize}

\pagebreak

kann den weißen Turm schlagen

\pagebreak

Schau dir die folgende Position an.
Entscheide, ob der schwarze König dem Schach entkommen kann.

\begin{center}
  \chessboard[setfen=6Rk/4N2p/8/8/8/8/8/6K]
\end{center}

Der schwarze König:
\begin{itemize}
  \item[$\square$] kann den weißen Turm schlagen
  \item[$\square$] kann den weißen Turm nicht schlagen
\end{itemize}

\pagebreak

kann den weißen Turm nicht schlagen

Der Springer deckt den Turm.

\pagebreak

Schau dir die folgende Position an.
Entscheide, ob der schwarze König dem Schach entkommen kann.

\begin{center}
  \chessboard[setfen=R/kp/1p/8/8/8/1K/8]
\end{center}

Der schwarze König:
\begin{itemize}
  \item[$\square$] kann den weißen Turm schlagen
  \item[$\square$] kann den weißen Turm nicht schlagen
\end{itemize}

\pagebreak

kann den weißen Turm schlagen

\pagebreak

Schau dir die folgende Position an.
Entscheide, ob der schwarze König dem Schach entkommen kann.

\begin{center}
  \chessboard[setfen=k1R/pp/8/8/8/4K/8/8]
\end{center}

Der schwarze König:
\begin{itemize}
  \item[$\square$] kann den weißen Turm schlagen
  \item[$\square$] kann den weißen Turm nicht schlagen
\end{itemize}

\pagebreak

kann den weißen Turm nicht schlagen

Schwarz ist matt, da der König sich nur ein Feld fortbewegen kann und der Turm
somit außer Reichweite für den König ist.

\pagebreak

Schau dir die folgende Position an.
Entscheide, was der schwarze König in dieser Position tun kann.

\begin{center}
  \chessboard[setfen=7k/7p/8/3B/3B/6K/8/8]
\end{center}

Der schwarze König:
\begin{itemize}
  \item[$\square$] kann den Angreifer schlagen
  \item[$\square$] kann flüchten
  \item[$\square$] kann nichts von beiden tun
\end{itemize}

\pagebreak

kann nichts von beiden tun

Die beiden Läufer sind zu gut aufgestellt.

\pagebreak

Schau dir die folgende Position an.
Entscheide, was der schwarze König in dieser Position tun kann.

\begin{center}
  \chessboard[setfen=4k2q/2pp1Q/8/1b2N/p/8/3P1PP/4K]
\end{center}

Der schwarze König:
\begin{itemize}
  \item[$\square$] kann den Angreifer schlagen
  \item[$\square$] kann flüchten
  \item[$\square$] kann nichts von beiden tun
\end{itemize}

\pagebreak

kann flüchten

Nur ein Feld bleibt übrig – welches?

\pagebreak

Schau dir die folgende Position an.
Entscheide, was der schwarze König in dieser Position tun kann.

\begin{center}
  \chessboard[setfen=4R1k/6p/6Pp/8/8/8/6K/8]
\end{center}

Der schwarze König:
\begin{itemize}
  \item[$\square$] kann den Angreifer schlagen
  \item[$\square$] kann flüchten
  \item[$\square$] kann nichts von beiden tun
\end{itemize}

\pagebreak

kann nichts von beiden tun

\pagebreak

Schau dir die folgende Position an.
Entscheide, was der schwarze König in dieser Position tun kann.

\begin{center}
  \chessboard[setfen=7k/4N1pR/8/8/8/8/3K/8]
\end{center}

Der schwarze König:
\begin{itemize}
  \item[$\square$] kann den Angreifer schlagen
  \item[$\square$] kann flüchten
  \item[$\square$] kann nichts von beiden tun
\end{itemize}

\pagebreak

kann den Angreifer schlagen

\pagebreak

Schau dir die folgende Position an.
Entscheide, was der schwarze König in dieser Position tun kann.

\begin{center}
  \chessboard[setfen=rk/1pP/1P2Q/8/8/8/8/7K]
\end{center}

Der schwarze König:
\begin{itemize}
  \item[$\square$] kann den Angreifer schlagen
  \item[$\square$] kann flüchten
  \item[$\square$] kann nichts von beiden tun
\end{itemize}

\pagebreak

kann nichts von beiden tun

\pagebreak

Schau dir die folgende Position an.

\begin{center}
  \chessboard[setfen=8/8/2P/3K1k/2R3p/2q/8/8/]
\end{center}

Schwarz ist am Zug. Welcher Zug führt zu schachmatt?

\pagebreak

\begin{center}
  \chessboard[
    pgfstyle=straightmove,
    markmoves={c3-e5},
    setfen=8/8/2P/3K1k/2R3p/2q/8/8/
  ]
\end{center}

\pagebreak

Schau dir die folgende Position an.

\begin{center}
  \chessboard[setfen=2R3k/8/6P/5N/6K/8/8/8/]
\end{center}

Der weiße Turm setzt den schwarzen König schach.
Der schwarze König ist matt, der er keine Möglichkeit zur Flucht hat.

Welche Figuren von Weiß decken die möglichen Fluchtfelder?
\begin{itemize}
  \item \textsf{f8} wird gedeckt durch den Turm
  \item \textsf{f7} wird gedeckt durch
  \item \textsf{g7} wird gedeckt durch
  \item \textsf{h7} wird gedeckt durch
  \item \textsf{h8} wird gedeckt durch
\end{itemize}

\pagebreak

\begin{itemize}
  \item \textsf{f8} wird gedeckt durch den Turm
  \item \textsf{f7} wird gedeckt durch den Bauern
  \item \textsf{g7} wird gedeckt durch den Springer
  \item \textsf{h7} wird gedeckt durch den Bauern
  \item \textsf{h8} wird gedeckt durch den Turm
\end{itemize}

\pagebreak

Schau dir die folgende Position an.

\begin{center}
  \chessboard[setfen=2B1k/6P/6K/Q2N/8/8/8/8/]
\end{center}

Welche Figuren von Weiß decken die möglichen Fluchtfelder?
\begin{itemize}
  \item \textsf{d8} wird gedeckt durch
  \item \textsf{d7} wird gedeckt durch
  \item \textsf{e7} wird gedeckt durch
  \item \textsf{f7} wird gedeckt durch
  \item \textsf{f8} wird gedeckt durch
\end{itemize}

Welche weiße Figur bietet dem schwarzen König Schach?

\pagebreak

\begin{itemize}
  \item \textsf{d8} wird gedeckt durch die Dame
  \item \textsf{d7} wird gedeckt durch den Läufer
  \item \textsf{e7} wird gedeckt durch den Springer
  \item \textsf{f7} wird gedeckt durch den König
  \item \textsf{f8} wird gedeckt durch den Bauern
\end{itemize}

Keine weiße Figur bietet dem schwarzen König Schach.
Schwarz hat keinen gültigen Zug mehr; daher liegt hier ein \textbf{Patt} vor.

\pagebreak

Schau dir die folgende Position an.

\begin{center}
  \chessboard[setfen=3r1k/1q1P2b/7Q/p3p2p/1p/1B3P/PPP/1K1R/]
\end{center}

Weiß ist am Zug. Welcher Zug führt zu schachmatt?

\pagebreak

\begin{center}
  \chessboard[
    pgfstyle=straightmove,
    markmoves={h6-d6},
    setfen=3r1k/1q1P2b/7Q/p3p2p/1p/1B3P/PPP/1K1R/
  ]
\end{center}

\pagebreak

Schau dir die folgende Position an.

\begin{center}
  \chessboard[setfen=Q2k/8/2P/1KB/8/8/8/8/]
\end{center}

Auf welches einzige Feld kann der schwarze König flüchten?

\pagebreak

\begin{center}
  \chessboard[
    pgfstyle=straightmove,
    markmoves={d8-c7},
    setfen=Q2k/8/2P/1KB/8/8/8/8/
  ]
\end{center}

\pagebreak

Schau dir die folgende Position an.

\begin{center}
  \chessboard[setfen=6k/8/4KN1P/8/8/B/8/8/]
\end{center}

Auf welches einzige Feld kann der schwarze König flüchten?

\pagebreak

\begin{center}
  \chessboard[
    pgfstyle=straightmove,
    markmoves={g8-h8},
    setfen=6k/8/4KN1P/8/8/B/8/8/
  ]
\end{center}

\pagebreak

Schau dir die folgende Position an.

\begin{center}
  \chessboard[setfen=8/8/8/2K4k/6R/8/8/7Q/]
\end{center}

Durch welchen Zug kann der schwarze König dem Schach entkommen?

\pagebreak

\begin{center}
  \chessboard[
    pgfstyle=straightmove,
    markmoves={h5-g4},
    setfen=8/8/8/2K4k/6R/8/8/7Q/
  ]
\end{center}

\pagebreak

Schau dir die folgende Position an.

\begin{center}
  \chessboard[setfen=K1q/1r/8/8/8/4k/8/8/]
\end{center}

Kann sich der weiße König durch Schlagen des Turms retten?
\begin{itemize}
  \item[$\square$] ja
  \item[$\square$] nein (schachmatt)
\end{itemize}

\pagebreak

nein (schachmatt)

Die schwarze Dame gibt Schach und deckt den Turm. Weiß ist schachmatt.

\pagebreak

Schau dir die folgenden Positionen an.

\begin{center}
  A \chessboard[setfen=7k/5K/8/8/8/7R/8/8/]
  \qquad
  B \chessboard[setfen=8/8/7k/5K/8/8/7R/8/]
\end{center}

In welchen Stellungen ist Schwarz schachmatt?
\begin{itemize}
  \item[$\square$] nur A
  \item[$\square$] nur B
  \item[$\square$] beide
  \item[$\square$] keine
\end{itemize}

\pagebreak

nur A

In Stellung B hat der schwarze König mit \textsf{g7} noch ein Fluchtfeld.

\pagebreak

\section{Eigene Spiele}

\subsection{Gegen Sophie, Lichess, 23. Januar 2019}%
\label{sub:gegen_sophie_lichess_23_januar_2019}

\newchessgame

\setchessboard{showmover=true}

Siehe auch \url{https://lichess.org/rrdGlwBD}. Ich spiele Weiß, Sophie Schwarz.
Ich eröffne mit \mainline{1. d4} und Sophie spielt \mainline{1... g5}. Sie
sagte, dass sie eher „intuitiv spielt“, was ich daran erkenne, dass sie ihren
Bauern ohne Deckung ins Feld schickt. Ihr Bauer wird direkt von meinem schwarzen
Läufer attackiert. Ihr Zug war nicht gut.

\chessboard[
  pgfstyle=straightmove,
  arrow=to,
  linewidth=0.5mm,
  markmoves={c1-g5},
]

Ich nehme mir den Bauern mit \mainline{2. Bg5} und Sophie spielt \mainline{2...
f6}, um meinen Läufer zu verscheuchen. Damit öffnet sie jedoch c7 für Attacken
auf den König. Meine Idee ist jetzt, dass ich den Läufer opfere und dafür den
Weg für meine Dame nach h5 freimache, damit sie von dort den König Matt setzen
kann.

Stattdessen hätte sie den Bauern aufgeben und woanders öffnen sollen.
Die Engine schlägt \variation{2... c5} vor.

\chessboard[
  pgfstyle=straightmove,
  arrow=to,
  linewidth=0.5mm,
  markmoves={d1-h5, h5-e8},
]

Daher spiele ich \mainline{3. e3} und Sophie tappt in die Falle und nimmt den
Läufer mit \mainline{3... fxg5}. Wie angekündigt muss ich nur noch \mainline{4.
Qh5#} für Schachmatt spielen.

\chessboard[showmover=false]

\subsection{Gegen Michi, Lichess, 25. Februar 2019}%
\label{sub:gegen_michi_lichess_}

\newchessgame{}
\setchessboard{showmover=true}

Siehe auch \url{https://lichess.org/SEdL55j4lYxt}. Michi spielt mit Weiß sein
geliebtes Queens Gambit und gehe darauf mit dem Queens Gambit Declined ein.

Nach \mainline{1. d4 d5 2. c4 e6} spielt er \mainline{3. e3}, was ich ein
bisschen zu passiv finde.

\chessboard[
  inverse=true,
]

Leider gibt es zu diesem Zug keine Theorie in der Wikipedia, also müssen wir
selbst nachdenken. Schauen wir uns mögliche Varianten an:
\begin{itemize}
  \item \variation{3... dxc4} lädt ein zu \variation{4. Bxc4} und dann hab ich
    nichts gewonnen.

    \chessboard[
      inverse=true,
      setfen=rnbkqbnr/ppp2ppp/4p/8/2BP/4P/PP3PPP/RNBQK1NR
    ]

\end{itemize}

\subsection{Gegen Egon, Rostocker Schachverein, 13. Juni 2019}

Ich bin Weiß.

\newchessgame{}

\mainline{
  1. d4 d5
  2. c4 e6
  3. Bf4 c6
  4. e3 Nf6
  5. cxd5 exd5
  6. Nc3 Bf5
  7. Bd3 Bg6
  8. Nf3 Bb4
  9. O-O O-O
  10. a3 Bxc3
  11. bxc3 Ne4
  12. Bxe4 Bxe4
  13. Ne5 f6
  14. Ng4 Nd7
  15. f3 Bg6
  16. Qd2 Qe7
  17. Rfe1 Rfe8
  18. Re2 Nb6
  19. a4 Nc4
  20. Qe1 Qd7
  21. h3 Re6
  22. e4 dxe4
  23. fxe4 Rae8
  24. Nf2 Qe7
  25. Rb1 b6
  26. Rb4 Na5
}

\chessboard{}

Stockfish sagt $-1.2$, sieht also nicht gut aus für mich.

\end{document}
