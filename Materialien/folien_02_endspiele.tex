\documentclass[
  aspectratio=1610,
  xcolor={dvipsnames},
]{beamer}

% chktex-file 1
% chktex-file 26

\usefonttheme{professionalfonts}
\usefonttheme[stillsansseriflarge, stillsansserifsmall]{serif}

\usepackage{amsmath, amssymb}
\usepackage{booktabs}

\usepackage[MnSymbol]{mathspec}
\setmathfont(Digits,Latin,Greek)[
  Numbers={Lining, Proportional}
]{Minion Pro}
\setmathrm{Minion Pro}

\defaultfontfeatures{
  Ligatures=TeX,
  Scale=MatchLowercase,
}
\setsansfont{Gill Sans}
\setmainfont{Minion Pro}
\setmonofont{Fira Mono}

\usepackage{xltxtra}
\usepackage{polyglossia}
\setdefaultlanguage{german}

\usepackage{hyperref}

\setbeamertemplate{navigation symbols}{}

\graphicspath{{./bilder/}}

% Schachkram:
% siehe: http://mirrors.ctan.org/macros/latex/contrib/chessboard/chessboard.pdf
\usepackage{chessboard}
\usepackage{xskak}

%\usepackage{showframe}

\newcommand{\wplogo}{\includegraphics[height=2ex]{enwiki.png}}

\author{Martin Darmüntzel}
\title{Schach für Anfänger}
\subtitle{Endspiele}
%\institute{Universität Rostock}
%\date{6. September 2019}

\begin{document}

\begin{frame}{Aufwärmaufgaben}
  Entscheide, ob der schwarze König dem Schach entkommen kann.

  \begin{columns}[c]
    \begin{column}{0.5\textwidth}
      \chessboard[
        setfen=7k/6pQ/8/5B2/8/8/8/4K3,
        showmover=false,
      ]
    \end{column}
    \begin{column}{0.4\textwidth}
      Die weiße Dame gibt Schach. Der schwarze König:
      \begin{itemize}
        \item[$\square$] kann schlagen
        \only<1>{\item[$\square$] kann nicht schlagen}
        \only<2>{\item[$\checkmark$] \textbf{kann nicht schlagen}}
      \end{itemize}
    \end{column}
  \end{columns}
\end{frame}

\begin{frame}{Aufwärmaufgaben}
  Entscheide, ob der schwarze König dem Schach entkommen kann.

  \begin{columns}[c]
    \begin{column}{0.5\textwidth}
      \chessboard[
        setfen=k/Q/8/8/2B/8/8/4K,
        showmover=false,
      ]
    \end{column}
    \begin{column}{0.4\textwidth}
      Die weiße Dame gibt Schach. Der schwarze König:
      \begin{itemize}
        \only<1>{\item[$\square$] kann schlagen}
        \only<2>{\item[$\checkmark$] \textbf{kann schlagen}}
        \item[$\square$] kann nicht schlagen
      \end{itemize}
    \end{column}
  \end{columns}
\end{frame}

\begin{frame}{Aufwärmaufgaben}
  Entscheide, ob der schwarze König dem Schach entkommen kann.

  \begin{columns}[c]
    \begin{column}{0.5\textwidth}
      \chessboard[
        setfen=6Rk/7p/8/5N/8/8/8/6K,
        showmover=false,
      ]
    \end{column}
    \begin{column}{0.4\textwidth}
      Der schwarze König:
      \begin{itemize}
        \only<1>{\item[$\square$] kann den weißen Turm schlagen}
        \only<2>{\item[$\checkmark$] \textbf{kann den weißen Turm schlagen}}
        \item[$\square$] kann den weißen Turm nicht schlagen
      \end{itemize}
    \end{column}
  \end{columns}
\end{frame}

\begin{frame}{Aufwärmaufgaben}
  Entscheide, ob der schwarze König dem Schach entkommen kann.

  \begin{columns}[c]
    \begin{column}{0.5\textwidth}
      \chessboard[
        setfen=6Rk/4N2p/8/8/8/8/8/6K,
        showmover=false,
      ]
    \end{column}
    \begin{column}{0.4\textwidth}
      Der schwarze König:
      \begin{itemize}
        \item[$\square$] kann den weißen Turm schlagen
        \only<1>{\item[$\square$] kann den weißen Turm nicht schlagen}
        \only<2>{\item[$\checkmark$] \textbf{kann den weißen Turm nicht schlagen}}
      \end{itemize}
    \end{column}
  \end{columns}
\end{frame}

\begin{frame}{Aufwärmaufgaben}
  Entscheide, ob der schwarze König dem Schach entkommen kann.

  \begin{columns}[c]
    \begin{column}{0.5\textwidth}
      \chessboard[
        setfen=R/kp/1p/8/8/8/1K/8,
        showmover=false,
      ]
    \end{column}
    \begin{column}{0.4\textwidth}
      Der schwarze König:
      \begin{itemize}
        \only<1>{\item[$\square$] kann den weißen Turm schlagen}
        \only<2>{\item[$\checkmark$] \textbf{kann den weißen Turm schlagen}}
        \item[$\square$] kann den weißen Turm nicht schlagen
      \end{itemize}
    \end{column}
  \end{columns}
\end{frame}

\begin{frame}{Aufwärmaufgaben}
  Entscheide, ob der schwarze König dem Schach entkommen kann.

  \begin{columns}[c]
    \begin{column}{0.5\textwidth}
      \chessboard[
        setfen=k1R/pp/8/8/8/4K/8/8,
        showmover=false,
      ]
    \end{column}
    \begin{column}{0.4\textwidth}
      Der schwarze König:
      \begin{itemize}
        \item[$\square$] kann den weißen Turm schlagen
        \only<1>{\item[$\square$] kann den weißen Turm nicht schlagen}
        \only<2>{\item[$\checkmark$] \textbf{kann den weißen Turm nicht schlagen}}
      \end{itemize}
    \end{column}
  \end{columns}
\end{frame}

\begin{frame}{Aufwärmaufgaben}
  Entscheide, was der schwarze König in dieser Position tun kann.

  \begin{columns}[c]
    \begin{column}{0.5\textwidth}
      \chessboard[
        setfen=7k/7p/8/3B/3B/6K/8/8,
        showmover=false,
      ]
    \end{column}
    \begin{column}{0.4\textwidth}
      Der schwarze König:
      \begin{itemize}
        \item[$\square$] kann den Angreifer schlagen
        \item[$\square$] kann flüchten
        \only<1>{\item[$\square$] kann nichts tun}
        \only<2>{\item[$\checkmark$] \textbf{kann nichts tun}}
      \end{itemize}
    \end{column}
  \end{columns}
\end{frame}

\begin{frame}{Aufwärmaufgaben}
  Entscheide, was der schwarze König in dieser Position tun kann.

  \begin{columns}[c]
    \begin{column}{0.5\textwidth}
      \chessboard[
        setfen=4k2q/2pp1Q/8/1b2N/p/8/3P1PP/4K,
        showmover=false,
      ]
    \end{column}
    \begin{column}{0.4\textwidth}
      Der schwarze König:
      \begin{itemize}
        \item[$\square$] kann den Angreifer schlagen
        \only<1>{\item[$\square$] kann flüchten}
        \only<2>{\item[$\checkmark$] \textbf{kann flüchten} – wohin?}
        \item[$\square$] kann nichts tun
      \end{itemize}
    \end{column}
  \end{columns}
\end{frame}

\begin{frame}{Aufwärmaufgaben}
  Entscheide, was der schwarze König in dieser Position tun kann.

  \begin{columns}[c]
    \begin{column}{0.5\textwidth}
      \chessboard[
        setfen=4R1k/6p/6Pp/8/8/8/6K/8,
        showmover=false,
      ]
    \end{column}
    \begin{column}{0.4\textwidth}
      Der schwarze König:
      \begin{itemize}
        \item[$\square$] kann den Angreifer schlagen
        \item[$\square$] kann flüchten
        \only<1>{\item[$\square$] kann nichts tun}
        \only<2>{\item[$\checkmark$] \textbf{kann nichts tun}}
      \end{itemize}
    \end{column}
  \end{columns}
\end{frame}

\begin{frame}{Aufwärmaufgaben}
  Entscheide, was der schwarze König in dieser Position tun kann.

  \begin{columns}[c]
    \begin{column}{0.5\textwidth}
      \chessboard[
        setfen=7k/4N1pR/8/8/8/8/3K/8,
        showmover=false,
      ]
    \end{column}
    \begin{column}{0.4\textwidth}
      Der schwarze König:
      \begin{itemize}
        \only<1>{\item[$\square$] kann den Angreifer schlagen}
        \only<2>{\item[$\checkmark$] \textbf{kann den Angreifer schlagen}}
        \item[$\square$] kann flüchten
        \item[$\square$] kann nichts tun
      \end{itemize}
    \end{column}
  \end{columns}
\end{frame}

\begin{frame}{Matt in 1}
  Finde den Mattzug. Für jeden zusätzlichen Mattzug gibt’s 1 Bonuspunkt.
\end{frame}

\begin{frame}{Nachtrag von letzter Woche}
  \begin{columns}[c]
    \begin{column}{0.5\textwidth}
      \only<1-1>{
        \chessboard[
          setfen=6Q1/1k4bR/2pP4/2B5/2B5/8/Rp6/K7 w - - 0 1, % chktex 8
          showmover=false,
        ]
      }
      \only<2-2>{
        \chessboard[
          setfen=6Q1/1k4bR/2pP4/2B5/2B5/8/1R/K7 w - - 0 1, % chktex 8
          showmover=false,
        ]
      }
      \only<3-3>{
        \chessboard[
          setfen=6Q1/1k4bR/2pP4/2B5/2B5/8/1R/K7 w - - 0 1, % chktex 8
          showmover=false,
          arrow=to,
          pgfstyle=straightmove,
          color=red,
          linewidth=2pt,
          opacity=0.8,
          markmoves={b2-b7,g7-a1},
        ]
      }
    \end{column}
    \begin{column}{0.4\textwidth}
      Eine Figur greift ein Feld an, auch wenn sie am Zug gehindert ist, weil sie
      andernfalls den eigenen König im Angriff stehen lassen oder ihn einem Angriff
      aussetzen würde.
    \end{column}
  \end{columns}
\end{frame}

\begin{frame}
  \titlepage{}
\end{frame}

\begin{frame}
  \begin{center}
    \begin{tabular}{ll}
      \WhiteKingOnWhite \WhiteQueenOnWhite \WhiteRookOnWhite & \BlackKingOnWhite\\
      \WhiteKingOnWhite \WhiteRookOnWhite \WhiteRookOnWhite & \BlackKingOnWhite\\
      \WhiteKingOnWhite \WhiteQueenOnWhite & \BlackKingOnWhite\\
      \WhiteKingOnWhite \WhiteRookOnWhite & \BlackKingOnWhite\\
      \WhiteKingOnWhite \WhiteBishopOnWhite \WhiteBishopOnWhite & \BlackKingOnWhite\\
      \WhiteKingOnWhite \WhiteRookOnWhite \WhiteBishopOnWhite \WhiteKnightOnWhite & \BlackKingOnWhite\\
      \WhiteKingOnWhite \WhiteQueenOnWhite & \BlackKingOnWhite \BlackBishopOnWhite\\
      \WhiteKingOnWhite \WhiteQueenOnWhite & \BlackKingOnWhite \BlackKnightOnWhite\\
      \WhiteKingOnWhite \WhitePawnOnWhite & \BlackKingOnWhite
    \end{tabular}
  \end{center}
\end{frame}

\begin{frame}
  \begin{center}
    \begin{tabular}{ll}
      \WhiteKingOnWhite \WhiteQueenOnWhite \WhiteRookOnWhite & \BlackKingOnWhite\\
      \WhiteKingOnWhite \WhiteRookOnWhite \WhiteRookOnWhite & \BlackKingOnWhite\\
    \end{tabular}
  \end{center}
\end{frame}

\setbeamercolor{frametitle}{fg=black}

\begin{frame}{\WhiteKingOnWhite \WhiteQueenOnWhite \WhiteRookOnWhite \BlackKingOnWhite}
  \begin{center}
    \chessboard[
      setfen=8/8/8/3k/8/8/8/5RQK w - - 0 1, % chktex 8
      moveid=1w,
    ]
  \end{center}
\end{frame}

\begin{frame}{\WhiteKingOnWhite \WhiteRookOnWhite \WhiteRookOnWhite \BlackKingOnWhite}
  \begin{center}
    \chessboard[
      setfen=8/8/8/8/8/4k/6R/6RK w - - 0 1, % chktex 8
      moveid=1w,
    ]
  \end{center}
\end{frame}

\begin{frame}{\WhiteKingOnWhite \WhiteQueenOnWhite \BlackKingOnWhite}
  \begin{center}
    \chessboard[
      setfen=8/8/8/3k/8/8/8/6QK w - - 0 1, % chktex 8
      moveid=1w,
    ]
  \end{center}
\end{frame}

\begin{frame}{\WhiteKingOnWhite \WhiteRookOnWhite \BlackKingOnWhite}
  \begin{center}
    \chessboard[
      setfen=8/8/8/3k/8/8/8/6RK w - - 0 1, % chktex 8
      moveid=1w,
    ]
  \end{center}
\end{frame}

\begin{frame}{\WhiteKingOnWhite \WhiteBishopOnWhite \WhiteBishopOnWhite \BlackKingOnWhite}
  \begin{center}
    \chessboard[
      setfen=8/8/8/3k/8/8/8/5BBK w - - 0 1, % chktex 8
      moveid=1w,
    ]
  \end{center}
\end{frame}

\begin{frame}{\WhiteKingOnWhite \WhiteRookOnWhite \WhiteBishopOnWhite \WhiteKnightOnWhite \BlackKingOnWhite}
  \begin{center}
    \chessboard[
      setfen=8/8/8/3k/8/8/8/4NBRK w - - 0 1, % chktex 8
      moveid=1w,
    ]

    Der weiße König darf nicht bewegt werden.
  \end{center}
\end{frame}

\begin{frame}{\WhiteKingOnWhite \WhiteQueenOnWhite \BlackKingOnWhite \BlackBishopOnWhite}
  \begin{center}
    \chessboard[
      setfen=8/8/3b/6k/8/8/2K/Q w - - 0 1, % chktex 8
      moveid=1w,
    ]
  \end{center}
\end{frame}

\begin{frame}{\WhiteKingOnWhite \WhiteQueenOnWhite \BlackKingOnWhite \BlackKnightOnWhite}
  \begin{center}
    \chessboard[
      setfen=8/8/8/3nk/8/3Q/8/7K w - - 0 1, % chktex 8
      moveid=1w,
    ]
  \end{center}
\end{frame}

\begin{frame}
  \begin{center}
    \begin{tabular}{ll}
      \WhiteKingOnWhite \WhiteQueenOnWhite \WhiteRookOnWhite & \BlackKingOnWhite\\
      \WhiteKingOnWhite \WhiteRookOnWhite \WhiteRookOnWhite & \BlackKingOnWhite\\
      \WhiteKingOnWhite \WhiteQueenOnWhite & \BlackKingOnWhite\\
      \WhiteKingOnWhite \WhiteRookOnWhite & \BlackKingOnWhite\\
      \WhiteKingOnWhite \WhiteBishopOnWhite \WhiteBishopOnWhite & \BlackKingOnWhite\\
      \WhiteKingOnWhite \WhiteRookOnWhite \WhiteBishopOnWhite \WhiteKnightOnWhite &
      \BlackKingOnWhite\\
      \WhiteKingOnWhite \WhiteQueenOnWhite & \BlackKingOnWhite \BlackBishopOnWhite\\
      \WhiteKingOnWhite \WhiteQueenOnWhite & \BlackKingOnWhite \BlackKnightOnWhite\\
      \WhiteKingOnWhite \WhitePawnOnWhite & \BlackKingOnWhite
    \end{tabular}
  \end{center}
\end{frame}

\begin{frame}{
    \WhiteKingOnWhite \WhiteQueenOnWhite \BlackKingOnWhite \BlackBishopOnWhite
    \hfill
    \WhiteKingOnWhite \WhiteQueenOnWhite \BlackKingOnWhite \BlackKnightOnWhite
  }
  \begin{columns}[c]
    \begin{column}{0.5\textwidth}
      \chessboard[
        setfen=8/8/3b/6k/8/8/2K/Q w - - 0 1, % chktex 8
        moveid=1w,
        showmover=false,
      ]
    \end{column}
    \begin{column}{0.5\textwidth}
      \chessboard[
        setfen=8/8/8/3nk/8/3Q/8/7K w - - 0 1, % chktex 8
        moveid=1w,
        showmover=false,
      ]
    \end{column}
  \end{columns}
\end{frame}

\begin{frame}{\WhiteKingOnWhite \WhitePawnOnWhite \BlackKingOnWhite}
  \begin{center}
    \chessboard[
      setfen=8/8/8/8/5P/k/8/K w - - 0 1 % chktex 8
      moveid=1w
      smallboard,
      showmover=false,
    ]
  \end{center}
\end{frame}

\begin{frame}{\WhiteKingOnWhite \WhitePawnOnWhite \BlackKingOnWhite}
  \begin{center}
    \chessboard[
      showmover=false,
      setfen=8/8/8/8/5P/k/8/K w - - 0 1, % chktex 8
      moveid=1w,
      padding=-0.8ex,
      pgfstyle={[fill]circle},
      markfields={b4, c5, d6, e7, f8},
      pgfstyle=color,
      padding=-0.3ex,
      opacity=0.3,
      color=Maroon,
      markregion={b4-f8},
    ]
  \end{center}
\end{frame}

\begin{frame}{\WhiteKingOnWhite \WhitePawnOnWhite \BlackKingOnWhite}
  \begin{center}
    \chessboard[
      setfen=8/8/8/8/6P/2k/8/K w - - 0 1, % chktex 8
      moveid=1w
      smallboard,
      showmover=true,
    ]

    Kann Weiß den Bauern noch befördern?
  \end{center}
\end{frame}

\begin{frame}{\WhiteKingOnWhite \WhitePawnOnWhite \BlackKingOnWhite}
  \begin{center}
    \chessboard[
      setfen=8/6k1/8/8/1P6/8/8/7K b - - 0 1, % chktex 8
      moveid=1b
      smallboard,
      showmover=true,
      inverse=true,
    ]

    Kann Schwarz den Bauern einholen?
  \end{center}
\end{frame}

\begin{frame}
  \begin{center}
    \chessboard[
      setfen=4k3/8/8/8/8/8/4P3/4K3 w - - 0 1, % chktex 8
      moveid=1w
      smallboard,
      showmover=false,
    ]
  \end{center}
\end{frame}

\begin{frame}
  \begin{center}
    \begin{tabular}{ll}
      \WhiteKingOnWhite \WhiteQueenOnWhite \WhiteRookOnWhite & \BlackKingOnWhite\\
      \WhiteKingOnWhite \WhiteRookOnWhite \WhiteRookOnWhite & \BlackKingOnWhite\\
      \WhiteKingOnWhite \WhiteQueenOnWhite & \BlackKingOnWhite\\
      \WhiteKingOnWhite \WhiteRookOnWhite & \BlackKingOnWhite\\
      \WhiteKingOnWhite \WhiteBishopOnWhite \WhiteBishopOnWhite & \BlackKingOnWhite\\
      \WhiteKingOnWhite \WhiteRookOnWhite \WhiteBishopOnWhite \WhiteKnightOnWhite &
      \BlackKingOnWhite\\
      \WhiteKingOnWhite \WhiteQueenOnWhite & \BlackKingOnWhite \BlackBishopOnWhite\\
      \WhiteKingOnWhite \WhiteQueenOnWhite & \BlackKingOnWhite \BlackKnightOnWhite\\
      \WhiteKingOnWhite \WhitePawnOnWhite & \BlackKingOnWhite
    \end{tabular}
  \end{center}
\end{frame}

\end{document}
