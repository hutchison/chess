\documentclass[
  a4paper,
  justified,
  nobib,
  ngerman,
]{tufte-handout}

% chktex-file 1
% chktex-file 11
% chktex-file 12

% Deutsche Sprache bei Silbentrennung und Datum:
\usepackage{babel}

\title{The Sensei Speedrun}
\author{Daniel Naroditsky}
%\publisher{Hochschulsport der Universität Rostock}
%\date{} % without \date command, current date is supplied

% page background color
% Comment out the following line if you do want to add a page background color (e.g. for
% printing)
% \pagecolor{LightChamois}

% Schachkram:
% siehe: http://mirrors.ctan.org/macros/latex/contrib/chessboard/chessboard.pdf
\usepackage{chessboard}
\usepackage{xskak}

\def\arrowcolor{orange!80}
\def\arrowlinewidth{2pt}
\def\arrowopacity{0.618}

\storechessboardstyle{standard}{%
  arrow=to,
  color=\arrowcolor,
  linewidth=\arrowlinewidth,
  opacity=\arrowopacity,
  pgfstyle=straightmove,
  shortenend=0.618ex,
  moverbottomlift=-3ex,
  movertoplift=-2ex,
  movershift=-1.618ex,
  marginleft=false,
  marginright=false,
  %marginrightwidth=0pt,
  %marginleftwidth=0pt,
  marginbottomwidth=1.5em,
}
\renewcommand\xskakcomment[1]{{\normalfont{#1}}}%

\usepackage{enumerate}

\newenvironment{notiz}{
  \color{Maroon}
  \paragraph*{Notiz}
}{
  \color{black}
}

\newcommand\kurznotiz[1]{\textcolor{Maroon}{#1}}

\usepackage{hyperref}

\usepackage{cleveref}

\begin{document}

\maketitle

Das Ziel dieses Speedruns soll sein, mit verständlichen Mitteln Partien zu gewinnen und
die Varianten und Methoden nachvollziehbar zu erklären.

\section{SenseiDanya (700) – marycp (965)}%
\label{sec:senseidanya_marycp}

\newchessgame

Zu Beginn wird auf jeden Fall \variation{1. e4} gespielt. Dabei wird das Schottische
Vierspringerspiel angestrebt (four knight scotch).

\mainline{1. e4 e5 2. Nf3 Qf6} zeigt jedoch, dass der Gegner anderes im Sinn hat
(\cref{fig:partie1_d1}).
\begin{marginfigure}
  \chessboard[style=standard, tinyboard]
  \caption{Diagramm 1}%
  \label{fig:partie1_d1}
\end{marginfigure}
Der letzte Zug von Schwarz ist aus mehreren Gründen schlecht: es wird keine Leichtfigur
entwickelt, die Dame wird zu früh rausgeholt und f6 wird für die Entwicklung des Springers
blockiert.

Wie kann dieser Zug ausgenutzt werden?
Zuerst wird ein Entwicklungsvorsprung herausgeholt und erst dann wird das Zentrum
geöffnet.
Für die Entwicklung gibt es mehrere Möglichkeiten: \variation{3. Nc3} oder \variation{3.
c3} mit der Idee \variation{4. d4}. Daniel spielt lieber \mainline{3. Nc3}.

Der Gegner antwortet mit \mainline{3... c6}. Daniel möchte jetzt doch schon das Zentrum
öffnen und entschließt sich für \mainline{4. d4} (\cref{fig:partie1_d2}).
\begin{marginfigure}
  \chessboard[style=standard, tinyboard]
  \caption{Diagramm 2}%
  \label{fig:partie1_d2}
\end{marginfigure}
Nach \mainline{4... exd4} gäbe es jetzt die Möglichkeit für den Zwischenzug \variation{5.
Bg5} um ein Tempo auf die Dame zu gewinnen, aber Daniel hält es einfach und nimmt mit dem
Springer zurück (\mainline{5. Nxd4}) worauf Schwarz diesen Springer mit dem Läufer
angreift: \mainline{5... Bc5} (\cref{fig:partie1_d3}).
\begin{marginfigure}
  \chessboard[style=standard, tinyboard]
  \caption{Diagramm 3}%
  \label{fig:partie1_d3}
\end{marginfigure}
Wie geht’s jetzt weiter? Den Springer verteidigen oder zurückziehen?

Der Springer kann mit dem Läufer verteidigt werden, sodass dieser gleich entwickelt wird.
Der Springer kann von Schwarz nicht weiter angegriffen werden und der Läufer selbst wird
durch einen Bauern verteidigt.
Es spricht also nichts gegen \mainline{6. Be3}.

Schwarz spielt \mainline{6... d6}, wahrscheinlich um den Läufer zu verteidigen, und Platz
für den anderen Läufer und den Springer zu machen.

Welche Figuren muss Weiß noch entwickeln?
Der König muss rochieren bzw. in Sicherheit gebracht werden.
Der Läufer von f1 muss entwickelt werden.
Die Dame und die Türme müssen auch ins Spiel kommen.

Wenn der Läufer entwickelt wird, wäre die Möglichkeit für kurze Rochade offen.
Tatsächlich spricht auch nichts gegen eine lange Rochade.
Wohin soll der Läufer?
Wenn er auf d3 geht, dann hängt der Springer – das geht also nicht.
Bleiben nur e2 und c4.
Auf c4 ist er aktiver, also \mainline{7. Bc4}.

Was machen wir, wenn \variation{7... b5} gespielt wird?
Dann kann ein Springer geopfert werden, denn z.\,B. nach \variation{8. Ncxb5 cxb5 9. Bd5}
gewinnt Weiß im nächsten Zug den Turm auf a8.

Daniel sagt eher: wenn \variation{7... b5} gespielt wird, ziehen wir den Läufer zurück
mittels \variation{8. Bb3}.

Schwarz spielt \mainline{7... Nh6} – ein ungewöhnlicher Entwicklungszug, da der Springer
an den Rand geht.
Was ist die Idee dahinter?
Es ist nicht die Entwicklung des Läufers auf g4 um die Dame anzugreifen – wir würden mit
der Dame einfach ausweichen.
Es ist viel mehr die Drohung von \variation{8... Ng4} um den Springer gegen den Läufer auf
e3 abzutauschen und die Bauernstruktur von Weiß zu zerstören.

Wie kann diese Idee verhindert werden?
Dafür gibt es zwei Möglichkeiten: \variation{8. f3} oder \variation{8. h3}. In 90\,\% der
Fälle ist \variation{8. h3} die richtige Entscheidung, daher wird das auch gespielt:
\mainline{8. h3} (\cref{fig:partie1_d4}).
\begin{marginfigure}
  \chessboard[style=standard, tinyboard]
  \caption{Diagramm 4}%
  \label{fig:partie1_d4}
\end{marginfigure}
Schwarz rochiert kurz: \mainline{8... O-O}.
Da Weiß schon einen Bauern auf der Königsseite gezogen hat (und damit eine mögliche
Schwäche induziert hat) und Harry der h-Bauer als Anker für den Vorstoß g4 dienen könnte,
ist es vielleicht besser lang zu rochieren.
Dies wird durch \mainline{9. Qd2} vorbereitet.

Schwarz greift den Läufer auf c4 mittels \mainline{9... b5} an.
Daniel erkennt eine mögliche taktische Kombination (siehe oben), sagt aber, dass es
ziemlich \emph{messy} würde und zieht den Läufer stattdessen mittels \mainline{10. Bb3}
zurück.
\begin{notiz}
  TODO: diese Taktik nochmal genauer aufdröseln. So eindeutig ist es nämlich nicht.
\end{notiz}

Schwarz entwickelt den Springer mittels \mainline{10... Na6} an den Rand und Weiß rochiert
– wie geplant – lang: \mainline{11. O-O-O} (\cref{fig:partie1_d5}).
Es wäre auch möglich gewesen einen Bauern mittels \variation{11. Nxc6} zu gewinnen.
\begin{marginfigure}
  \chessboard[style=standard, tinyboard]
  \caption{Diagramm 5}%
  \label{fig:partie1_d5}
\end{marginfigure}
Nach der Rochade sind alle Figuren entwickelt und durch die gegenüberliegene Rochade kann
jetzt der Angriff auf dem Königsflügel gestartet werden.

Schwarz spielt \mainline{11... Bb6} um Platz für weitere Bauernvorstöße zu machen.
Weiß schnappt sich den ungedeckten Bauern \mainline{12. Nxc6}.
Dadurch hängen bei Schwarz die Bauern auf b5 und d6.
Schwarz spielt den Springer näher zum Zentrum mittels \mainline{12... Nc5}
(\cref{fig:partie1_d6}).
\begin{marginfigure}
  \chessboard[style=standard, tinyboard, pgfstyle=circle, color=red, backfields={b5,d6}]
  \caption{Diagramm 6}%
  \label{fig:partie1_d6}
\end{marginfigure}

Weiß könnte jetzt einen der beiden Bauern nehmen, aber schauen wir uns die Konsequenzen
an:
\begin{itemize}
  \item Nach \variation{13. Nxb5} wird der Springer vom Zentrum wegbewegt und der Bauer
    auf e4 hängt, sodass Schwarz mit \variation{13... Nxe4} gleichzeitig die Dame angreift
    und den Bauern auf d6 verteidigt.

    \begin{center}
      \chessboard[
        style=standard,
        smallboard,
        setfen=r1b2rk1/p4ppp/1bNp1q1n/1N6/4n3/1B2B2P/PPPQ1PP1/2KR3R w - - 0 14,
        backmoves={e4-d2, e4-d6},
      ]
    \end{center}

  \item Nach \variation{13. Qxd6 Qxd6 14. Rxd6} könnte Schwarz alle Figuren abtauschen
    (\variation{14... Nxb3 15. axb3 Bxe3 16. fxe3}) und würde Weiß mit zwei
    Dop"-pel"-bau"-ern das Endspiel schwer machen.

    Das ist kein Weltuntergang und die Stellung ist immer noch gut für Weiß, aber es nervt
    doch.

    \begin{center}
      \chessboard[
        style=standard,
        smallboard,
        setfen=r1b2rk1/p4ppp/2NR3n/1p6/4P3/1PN1P2P/1PP3P1/2K4R b - - 0 16,
        pgfstyle=circle,
        backfields={b2,b3,e3,e4},
      ]
    \end{center}
\end{itemize}

Gibt es noch einen besseren Zug?
Der Zug \mainline{13. Nd5} bringt den Springer ins Zentrum und greift die Dame an.
Falls die Dame auf g6 ausweicht, könnte Weiß eine Springergabel auf e7 ansetzen.
Auf e6 könnte die Dame mittels \variation{14. Nde7+} durch einen Abzugsangriff gewonnen
werden.
Schwarz müsste jetzt zunächste mit dem Springer den Läufer auf b3 nehmen und dann die Dame
auf e6 setzen.

Stattdessen spielt Schwarz \mainline{13... Qg6} und läuft direkt in die Gabel hinein.
Es ist ziemlich egal, mit welchem Springer die Gabel ausgeführt wird, daher spielt Daniel
\mainline{14. Nce7+ Kh8 15. Nxg6+ fxg6} (\cref{fig:partie1_d7}).
\begin{marginfigure}
  \chessboard[style=standard, tinyboard]
  \caption{Diagramm 7}%
  \label{fig:partie1_d7}
\end{marginfigure}

Wie kann Weiß mit diesem Materialvorteil die Partie sicher zum Sieg führen?
Prinzipiell gilt: zerstöre die Bauernstruktur des Gegners und damit die Stellung des
Gegners.

Hier können wir jedoch etwas geschickter an die Sache gehen.
Der schwarze König hat sich hinter den Bauern versteckt.
Können wir diesen Schutz öffnen?
Ja, mit \mainline{16. Bxh6 gxh6}.
Damit wird die lange Diagonale frei und es ist ein Matt in 3 (\cref{fig:partie1_d8}).
\begin{marginfigure}
  \chessboard[style=standard, tinyboard]
  \caption{Diagramm 8, Matt in 3}%
  \label{fig:partie1_d8}
\end{marginfigure}
Entweder \variation{17. Qc3+} oder \variation{17. Qd4+} und dann jeweils
\variation{17... Rf6 18. Qxf6+ Kg8 19. Ne7+#}.

In der Partie wurde jedoch \mainline{17. Qc3+ Kg8 18. Ne7#} gespielt.

\end{document}
