\documentclass[
  a4paper,
  justified,
  nobib,
]{tufte-handout}

% chktex-file 1
% chktex-file 11
% chktex-file 12

% Deutsche Sprache bei Silbentrennung und Datum:
\usepackage[ngerman]{babel}

\title{Logisches Schach – Zug für Zug}
\author{Original von Irving Chernev – übersetzt von Martin Darmüntzel}
%\publisher{Hochschulsport der Universität Rostock}
%\date{} % without \date command, current date is supplied

% page background color
% Comment out the following line if you do want to add a page background color (e.g. for
% printing)
% \pagecolor{LightChamois}

% Schachkram:
% siehe: http://mirrors.ctan.org/macros/latex/contrib/chessboard/chessboard.pdf
\usepackage{chessboard}
\usepackage{xskak}

\def\arrowcolor{orange!80}
\def\arrowlinewidth{2pt}
\def\arrowopacity{0.618}

\storechessboardstyle{standard}{%
  arrow=to,
  color=\arrowcolor,
  linewidth=\arrowlinewidth,
  opacity=\arrowopacity,
  pgfstyle=straightmove,
  shortenend=0.618ex,
  moverbottomlift=-3ex,
  movertoplift=-2ex,
  movershift=-1.618ex,
  marginleft=false,
  marginright=false,
  %marginrightwidth=0pt,
  %marginleftwidth=0pt,
  marginbottomwidth=1.5em,
}
\renewcommand\xskakcomment[1]{{\normalfont{#1}}}%

\usepackage{enumerate}

\newenvironment{notiz}{
  \color{Maroon}
  \paragraph*{Notiz}
}{
  \color{black}
}

\newcommand\kurznotiz[1]{\textcolor{Maroon}{#1}}

\usepackage{hyperref}

\begin{document}

\maketitle

\section{Spiel 1: von Scheve – Teichmann, Berlin 1907, Giuoco Piano}%
\label{sec:spiel_1_von_scheve_teichmann}

Die Hauptaufgabe in der Eröffnung lautet: bringe die Figuren ins Spiel – weg von der
letzten Reihe hinein in das aktive Spiel.

Kein Angriff (schon gar keine Mattkombination) funktioniert mit einer oder zwei Figuren.

Alle Figuren müssen entwickelt werden und jede hat eine Aufgabe zu erfüllen.

Für den Anfang ist es eine gute Idee gleich zwei Figuren auf einmal freizulassen.
Dies kann durch das Vorrücken eines Zentrumsbauern geschehen:

\newchessgame
\mainline{1. e4}

\begin{marginfigure}
  \chessboard[
    style=standard,
    smallboard,
  ]
\end{marginfigure}

Dies ist ein exzellenter Eröffnungszug.
Weiß verankert einen Bauern im Zentrum und öffnet Linien für Dame und Läufer.
Der nächste Zug, sofern er erlaubt wird, wird \variation{2. d4} sein.
Die beiden Bauern werden dann vier Felder auf der fünften Reihe (c5, d5, e5 und f5)
kontrollieren und Schwarz daran hindern Figuren auf diese wichtigen Felder zu setzen.

Wie soll Schwarz auf diesen ersten Zug von Weiß reagieren? Er darf keine Zeit mit
bedeutungslosen Zügen wie \variation{1... h6} oder \variation{1... a6} verschwenden.
Diese und andere ziellose Züge tragen nichts zur Entwicklung der Figuren bei und
behindern auch nicht die von Weiß angedrohte Bedrohung der Besetzung des Zentrums.

\emph{Schwarz muss um einen gleichen Anteil der wichtigen Felder im Zentrum kämpfen.
Schwarz muss die Besetzung des Zentrums erstreiten.}

Wozu der ganze Stress um das Zentrum? Warum ist es so wichtig?

Figuren, die im Zentrum stehen, haben die höchste Aktionsfreiheit und können ihre
Angriffkraft am besten ausspielen.
Ein Springer beispielsweise kann aus dem Zentrum heraus acht Felder angreifen, während er
am Rand auf vier Felder beschränkt ist.
Er ist nur ein halber Springer!

Die Besetzung des Zentrums bedeutet die Kontrolle über das wertvollste Gebiet.
Es lässt den gegnerischen Figuren weniger Raum und erschwert die Verteidigung, da sich die
Figuren gegenseitig in die Quere kommen.

Die Besetzung des Zentrums bzw. die Kontrolle aus der Entfernung baut eine Barriere auf,
die die gegnerischen Kräfte bei der harmonischen Zusammenarbeit stört.
Der Widerstand einer gespaltenen Armee ist normalerweise nicht sehr effektiv.

\mainline{1... e5}

Sehr gut!
Schwarz besteht auf einer fairen Aufteilung des Zentrums.
Er fixiert dort einen Bauern und befreit zwei Figuren.

\mainline{2. Nf3}

\begin{marginfigure}
  \chessboard[
    style=standard,
    smallboard,
  ]
\end{marginfigure}

Der beste Zug auf dem Brett!

Der Springer wird mit einer Drohung entwickelt – der Angriff auf den Bauern.
Das gewinnt Zeit, da Schwarz nicht so frei entwickeln kann, wie er möchte.
\emph{Er muss zuerst den Bauern retten} und das schränkt die Möglichkeiten der eigenen
Züge ein.

Der Springer wird \emph{zum Zentrum hin} entwickelt, was den Spielraum des Angriffs
erweitert.

\end{document}
