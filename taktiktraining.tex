\documentclass[
%% alle weiteren Papierformat einstellbar:
a5paper, %apaper,
%% keine Seitenzahlen:
%empty,
%% Schriftgröße (12pt, 11pt (Standard)):
11pt,
%% kleinere Überschriften:
%smallheadings,
]
{scrartcl}

% chktex-file 8 Wrong length of dash may have been used.
% chktex-file 11 You should use \ldots to achieve an ellipsis.
% chktex-file 24 Delete this space to maintain correct pagereferences.

\input{packages.latex}

\def\arrowcolor{orange!80}
\def\arrowlinewidth{2pt}
\def\arrowopacity{0.618}

\storechessboardstyle{standard}{%
  arrow=to,
  color=\arrowcolor,
  linewidth=\arrowlinewidth,
  opacity=\arrowopacity,
  pgfstyle=straightmove,
  shortenend=0.618ex,
}

\title{Taktiktraining}
\author{Martin Darmüntzel via \href{https://lichess.org}{lichess.org}}

\begin{document}

\maketitle

Was ist der beste Zug?

Siehe auch \url{https://de.wikipedia.org/wiki/Taktik_(Schach)}. % chktex 36

\pagebreak

Folgende Fragen können bei der Suche helfen:
\begin{itemize}
  \item Gibt es eine erzwungene Mattsequenz? Überprüfe die schachgebenden Züge.
    Überprüfe Züge, die dem König die Fluchtfelder nehmen.
  \item Kannst du ein Matt androhen, was zu Materialgewinn führt?
  \item Gibt es ungedeckte Figuren?
  \item Kann ein Springer gabeln?
  \item Gibt es überladene Figuren?
  \item Gibt es eine Fesselung (Pin), die man ausnutzen kann?
\end{itemize}

\pagebreak % neues Puzzle

\begin{center}
  \newchessgame[
    setfen=r4b1r/pppqk1p1/3p1nQ1/3Pp1Bp/2P3bP/5NP1/PP1N1P2/2KR3R b - - 2 14,
    moveid=14b
  ]
  \chessboard[
    style=standard,
    markmoves={e1-c1},
    inverse=true,
  ]
\end{center}

\index{Damenfang}

\pagebreak

\mainline{14... Bf5} greift die Dame an und sperrt sie ein.
Nach \mainline{15. Qxf5} nimmt Schwarz die Dame mit \mainline{15... Qxf5} und
Weiß kann nichts dagegen tun.
\begin{center}
  \chessboard[
    style=standard,
    inverse=true,
  ]
\end{center}

Taktische Motive: Damenfang

Quelle: \url{https://lichess.org/AuATzwQH/black#27}

\pagebreak % neues Puzzle

\begin{center}
  \newchessgame[
    setfen=2r1r1k1/3n1p2/ppnP2pp/2pB4/P2bN3/7P/5PPB/3RR1K1 w - - 4 27,
    moveid=27w,
  ]
  \chessboard[
    style=standard,
    markmoves={a8-c8},
    inverse=false,
  ]
\end{center}

\index{Abtausch}
\index{Öffnung}
\index{Opfer}
\index{Scheinopfer}

\pagebreak

\mainline{27. Bxc6 Rxc6} ist ein einfacher Springer-gegen-Läufer-Abtausch.
Darauf folgt \mainline{28. Nf6+} und Schwarz hat folgende Möglichkeiten:
\begin{itemize}
  \item \variation{28... Nxf6}, sodass Weiß mit \variation{29. Rxe8} den Turm
    nimmt und mehr Material hat (2 Türme, 1 Läufer, 5 Bauern [18] gegen 1 Turm,
    1 Läufer, 1 Springer, 6 Bauern [17]).
  \item \variation{28... Bxf6} führt auch zu \variation{29. Rxe8} mit
    Materialvorteil.
  \item Den König bewegen (z.\,B. \variation{28... Kf8} führt sogar zu
    \variation{29. Rxe8+}). Einzig möglicher Zug ist \variation{29. Kg7}, sodass
    Weiß sich den Springer mit \variation{29... Nxd7} schnappt.
\end{itemize}

\begin{center}
  \chessboard[
    style=standard,
    inverse=false,
  ]
\end{center}

Taktische Motive: Abtausch, Öffnung, Opfer, Scheinopfer

Quelle: \url{https://lichess.org/NDzBwBqn/white#52}

\pagebreak % neues Puzzle

\begin{center}
  \newchessgame[
    setfen=r2qkb2/pp1n1pp1/2pPp1b1/6pr/3P1n1P/5N2/PPP1QP1B/RN2KBR1 w Qq - 1 14,
    moveid=14w,
  ]
  \chessboard[
    style=standard,
    markmoves={d5-f4},
    inverse=false,
  ]
\end{center}

\index{Fesselung}
\index{Öffnung}

\pagebreak

Die weiße Dame auf e2 fesselt den schwarzen Bauern auf e6. Gleichzeitig greift
der schwarze Springer auf f4 die weiße Dame an. Das muss verhindert werden.
Welche Möglichkeiten hat die Dame um auszuweichen ohne dabei die Verteidigung
des Bauern auf c2 aufzugeben? \variation{14. Qc4} führt nur zu \variation{14...
b5} und scheucht die Dame durch die Gegend.

Also wird der Springer mit dem Läufer auf h2 genommen und Schwarz schlägt mit
dem Bauern zurück: \mainline{14. Bf4 gxf4}.

\begin{center}
  \chessboard[
    style=standard,
    inverse=false,
  ]
\end{center}

Damit öffnet Schwarz den Angriff des Turms auf g1 auf den Läufer auf g6.
Der Läufer wird nur durch den Bauern auf f7 verteidigt, welcher aber indirekt
durch die Dame gefesselt ist.
Daher \mainline{15. Rxg6} und Weiß hat mehr Material.

Taktische Motive: Fesselung, Öffnung

Quelle: \url{https://lichess.org/d6nNxsVG#26}

\pagebreak % neues Puzzle

\begin{center}
  \newchessgame[
    setfen=2krr3/ppp2ppp/3q4/3P3b/QnP1n3/4BN1P/P3BPP1/R4RK1 b - - 2 16,
    moveid=16b,
  ]
  \chessboard[
    style=standard,
    markmoves={c2-a4},
    inverse=true,
  ]
\end{center}

\index{Gabel}

\pagebreak

Die weiße Dame und der weiße Läufer sind beide ungedeckt. \mainline{16... Nc3}
gabelt beide. Weiß versucht den ursprünglichen Plan von \mainline{17. Qa7} zu
verfolgen, aber \mainline{17... Nxe2} holt Material und die Initiative.

\begin{center}
  \chessboard[
    style=standard,
    inverse=true,
  ]
\end{center}

Taktische Motive: Gabel

Quelle: \url{https://lichess.org/DhAfdSvP#31}

\pagebreak % neues Puzzle

\begin{center}
  \newchessgame[
    setfen=r6r/2p4k/ppnqN1pb/3pp1p1/1P6/P1PP3Q/1BB2PK1/R7 w - - 2 25,
    moveid=25w,
  ]
  \chessboard[
    style=standard,
    markmoves={f8-h8},
    inverse=false,
  ]
\end{center}

\index{Matt}
\index{Opfer}

\pagebreak

„When in doubt, sac the Queen!“

\mainline{25. Qxh6+} und Schwarz hat nur zwei mögliche Züge:
\begin{itemize}
  \item \variation{25... Kg8} und Weiß setzt Matt mit \variation{26. Qg7#}.
  \item \mainline{25... Kxh6} und Weiß setzt Matt mit \mainline{26. Rh1#}.
\end{itemize}

Matt in 2 Zügen.

\begin{center}
  \chessboard[
    style=standard,
    inverse=false,
  ]
\end{center}

Taktische Motive: Matt, Opfer

Quelle: \url{https://lichess.org/XIRrUvb9/white#48}

\pagebreak % neues Puzzle

\begin{center}
  \newchessgame[
    setfen=1k1r4/1p6/1P1p1ppp/2p1r3/4P2b/1B1K1P2/2P2q2/R4Q2 w - - 8 38,
    moveid=38w,
  ]
  \chessboard[
    style=standard,
    markmoves={h2-f2},
    inverse=false,
  ]
\end{center}

\index{Matt}
\index{Opfer}

\pagebreak

Halt dich fest: Matt in 6.

\mainline{
  38. Ra8+ Kxa8
  39. Qa1+ Kb8
  40. Qa7+ Kc8
  41. Qa8+ Kd7
  42. Qb7+ Ke8
  43. Qf7#
}

\begin{center}
  \chessboard[
    style=standard,
    inverse=false,
  ]
\end{center}

Taktische Motive: Matt, Opfer

Quelle: \url{https://lichess.org/guvoEh7j/white#74}

\pagebreak % neues Puzzle

\begin{center}
  \newchessgame[
    setfen=3r3r/1pp2qk1/p4n2/3PN1p1/1PPQP1P1/5pK1/5P2/3R3R b - - 2 31,
    moveid=31b,
  ]
  \chessboard[
    style=standard,
    markmoves={c3-d4},
    inverse=true,
  ]
\end{center}

\index{Matt}
\index{Abzug}

\pagebreak

Halt dich fest: Matt in 4. Wir starten mit
\mainline{
  31... Nh5+
}

Weiß hat nur drei mögliche Züge und alle führen zu Matt.
\begin{itemize}
  \item Weicht der König mit \variation{32. Kh3} aus, dann wird der Springer mit
    \variation{32... Ng3+} mit Schach durch den Turm abgezogen. Der König kann
    nur noch den Springer schlagen (\variation{33. Kxg3}) und Schwarz setzt Matt
    mit \variation{33... Qf4#}.
  \item Weicht der König mit \variation{32. Kh2} aus, dann wird der Springer
    auch hier mit \variation{32... Ng3+} mit Schach durch den Turm abgezogen.
    Der König kann entweder ausweichen oder schlagen.
    \begin{itemize}
      \item Weicht der König mit \variation{33. Kg1} aus, dann folgt Matt mit
        \variation{33... Rh1#}.
      \item Schlägt der König den Springer mit \variation{33. Kxg3}, dann folgt
        wieder Matt durch die Dame mit \variation{33... Qf4#}.
    \end{itemize}
  \item Der beste Zug für Weiß ist daher \mainline{32. Rxh5}. Schwarz muss nun
    auf die Dame aufpassen und setzt sie mit Schachgebot auf \mainline{32...
    Qf4+}. Einzig möglicher Zug für Weiß ist \mainline{33. Kh3} und Schwarz legt
    nach mit \mainline{33... Rxh5+}.
    Dies erzwingt \mainline{34. gxh5} % chktex 12
    und Schwarz setzt Matt mit \mainline{34... Qh4#}.
\end{itemize}

\begin{center}
  \chessboard[
    style=standard,
    smallboard,
    inverse=true,
  ]
\end{center}

Taktische Motive: Matt, Abzug

Quelle: \url{https://lichess.org/Kjr6x9qz/black#61} % chktex 29

\pagebreak % neues Puzzle

\begin{center}
  \newchessgame[
    setfen=r2q3r/4p1k1/pn4B1/3bP3/2p5/2N4R/PP6/2K3R1 w - - 1 29,
    moveid=29w,
  ]
  \chessboard[
    style=standard,
    markmoves={f8-h8},
    inverse=false,
  ]
\end{center}

\index{Matt}
\index{Abzug}

\pagebreak

Matt in 3.

Um die möglichen Fluchtfelder so weit wie möglich einzuschränken, ziehen wir den
Läufer auf \mainline{29. Be8+}, sodass Schwarz, um das Matt zu verzögern, nur
noch den eigenen Läufer mit \mainline{29... Bg2} dazwischenwerfen kann.
Dieser wird durch den Turm mit \mainline{30. Rxg2+} geschlagen und der König
flieht mit \mainline{30... Kf8}. Weiß setzt Matt mit \mainline{31. Rh8#}.

\begin{center}
  \chessboard[
    style=standard,
    inverse=false,
  ]
\end{center}

Taktische Motive: Matt, Abzug

Quelle: \url{https://lichess.org/Ceg0A7Dp/white#56}

\pagebreak % neues Puzzle

\begin{center}
  \newchessgame[
    setfen=r4rk1/7p/p1p1p3/PpPn1pqp/3P1P1R/2PQ2P1/2B3K1/3R4 b - - 0 34,
    moveid=34b,
  ]
  \chessboard[
    style=standard,
    markmoves={f3-f4},
    inverse=true,
  ]
\end{center}

\index{Überlastung}
\index{Gabel}
\index{Opfer}

\pagebreak

Der Springer auf d5 möchte den König und die Dame mit einem Sprung auf f4 gabeln, aber das
Feld ist durch den Bauern auf g3 gedeckt.
Dieser Bauer ist aber überladen, da er den Turm auf h4 und den Bauern auf f4 verteidigen
muss.

Mit \mainline{34... Qxh4} lenken wir den verteidigenden Bauern um (\mainline{35.
gxh4}) und ermöglichen die Gabel durch den Springer: \mainline{35... Nxf4}.
Der König weicht z.B. mit \mainline{36. Kf2} aus und wir nehmen die Dame mit
\mainline{36... Nxd3}.

\begin{center}
  \chessboard[
    inverse=true,
  ]
\end{center}

Taktische Motive: Überlastung, Gabel, Opfer

Quelle: \url{https://lichess.org/Nh0rpDWV/black#67}

\pagebreak % neues Puzzle

\begin{center}
  \newchessgame[
    setfen=1r3r1k/7p/pp1p2p1/2p3P1/P3q1P1/2P3NK/1P1Q3P/4nR2 w - - 0 28,
    moveid=28w,
  ]
  \chessboard[
    style=standard,
    markmoves={d3-e1},
    inverse=false,
  ]
\end{center}

\index{Abtausch}
\index{Zwischenzug}

\pagebreak

Wären wir gierig, dann würden wir die Dame auf e4 sofort mit dem Springer auf g3 nehmen.
Dann schlägt Schwarz jedoch unseren Turm auf f1 und der Computer schätzt die Situation als
gleich ein.

Es ist daher besser, wenn wir mit \mainline{28. Rxf8+} den Turm auf f8 mit Schachgebot
schlagen, sodass Schwarz mit \mainline{28... Rxf8} zurückschlägt und wir dann die Dame
mittels \mainline{29. Nxe4} schlagen. Dann hat Weiß signifikant mehr Material.

\begin{center}
  \chessboard[
    inverse=false,
  ]
\end{center}

Taktische Motive: Abtausch, Zwischenzug

Quelle: \url{https://lichess.org/k6sOHsO7/white#54}

\pagebreak % neues Puzzle

\begin{center}
  \newchessgame[
    setfen=rnb1k2r/p1p2ppp/1p2pn2/6B1/1bpqP3/2N2P2/PPQ3PP/R3KBNR w KQkq - 0 8,
    moveid=8w,
  ]
  \chessboard[
    style=standard,
    markmoves={d8-d4},
    inverse=false,
  ]
\end{center}

\index{Zwischenzug}

\pagebreak

Kann ich Schach geben und dann eine Figur gewinnen? Ja!

\mainline{8. Qa4+} und Schwarz hat verschiedene Möglichkeiten, um das Schachgebot zu
verteidigen, jedoch deckt keine davon den Läufer auf b4, sodass dieser im nächsten Zug
geschlagen werden kann (z.\,B. über \mainline{8... Qd7 9. Qxb4}).

\begin{center}
  \chessboard[
    inverse=false,
  ]
\end{center}

Taktische Motive: Zwischenzug

Quelle: \url{https://lichess.org/ViYrJQ6A/white#14}

\pagebreak % neues Puzzle

\begin{center}
  \newchessgame[
    setfen=8/4P1kp/5bp1/8/5P2/1pr5/4Q1PP/6K1 b - - 0 44,
    moveid=44b,
  ]
  \chessboard[
    style=standard,
    markmoves={e6-e7},
    inverse=true,
  ]
\end{center}

\index{Damenfang}
\index{Zwischenzug}

\pagebreak

Es gibt zwei Möglichkeiten um Weiß Schach zu geben: \variation{44... Rc1} oder
\variation{44... Bd4}.

\begin{itemize}
  \item bei \variation{44... Rc1} kann sich der König auf mit \variation{45. Kf2}
    verstecken und mit \variation{45... Bd4 46. Kf3} bleiben Schwarz zwar die Züge
    \begin{itemize}
      \item \variation{46... Rc2} um die Dame in Bedrängnis zu bringen,
      \item \variation{46... Rc3} um den König weiter zu scheuchen oder
      \item \variation{46... b2} um eine Umwandlung herbeizuführen. % chktex 12
    \end{itemize}
    Jedoch reicht nichts davon aus, um den König Matt zu setzen oder Material zu gewinnen.
\end{itemize}

Schauen wir uns daher den Läuferzug an: \mainline{44... Bd4} wird am besten durch die
Flucht mit \mainline{45. Kf1} pariert. Schwarz gibt Schach durch \mainline{45... Rc1} und
Weiß kann nur mit der Dame blockieren: \mainline{46. Qe1} und Schwarz schnappt sie sich
mit \mainline{46... Rxe1}. Danach ist genug Zeit um den Bauern auf e7 mit dem König zu
holen.

\begin{center}
  \chessboard[
    inverse=true,
  ]
\end{center}

Taktische Motive: Damenfang, Zwischenzug

Quelle: \url{https://lichess.org/zN2hPmc9/black#87}

\pagebreak

https://lichess.org/training/95056

Matt in 5

\pagebreak

https://lichess.org/training/95287

Forks forks forks!

\pagebreak

https://lichess.org/training/95288

Matt in 3, mit Turmopfer

\pagebreak

https://lichess.org/training/95261

En passant!

\pagebreak

https://lichess.org/training/95308

Fork!

\pagebreak

https://lichess.org/training/95309

Schlichter Materialgewinn durch die Bedrohung der Dame. Aber beim falschen Zug droht Matt
in 2.

\printindex

\end{document}
