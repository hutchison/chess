\documentclass[
  a4paper,
  justified,
  nobib,
]{tufte-handout}

% chktex-file 1
% chktex-file 11
% chktex-file 12

% Deutsche Sprache bei Silbentrennung und Datum:
\usepackage[ngerman]{babel}

\title{Schach}
\subtitle{Konzept des Sportkurses}
\author{Martin Darmüntzel}
%\publisher{Hochschulsport der Universität Rostock}
%\date{} % without \date command, current date is supplied

% page background color
% Comment out the following line if you do want to add a page background color (e.g. for
% printing)
% \pagecolor{LightChamois}

% Schachkram:
% siehe: http://mirrors.ctan.org/macros/latex/contrib/chessboard/chessboard.pdf
\usepackage{chessboard}
\usepackage{xskak}

\def\arrowcolor{orange!80}
\def\arrowlinewidth{2pt}
\def\arrowopacity{0.618}

\storechessboardstyle{standard}{%
  arrow=to,
  color=\arrowcolor,
  linewidth=\arrowlinewidth,
  opacity=\arrowopacity,
  pgfstyle=straightmove,
  shortenend=0.618ex,
  moverbottomlift=-3ex,
  movertoplift=-2ex,
  movershift=-1.618ex,
  marginleft=false,
  marginright=false,
  %marginrightwidth=0pt,
  %marginleftwidth=0pt,
  marginbottomwidth=1.5em,
}
\renewcommand\xskakcomment[1]{{\normalfont{#1}}}%

\usepackage{enumerate}

\newenvironment{notiz}{
  \color{Maroon}
  \paragraph*{Notiz}
}{
  \color{black}
}

\usepackage{hyperref}

\begin{document}

\maketitle

\section{Ziele}%
\label{sec:ziele}

\begin{itemize}
  \item gemeinsame Lernerfahrung
  \begin{itemize}
    \item niemand ist dumm, alle sind dumm
  \end{itemize}
  \item Regeln
  \item Endspiele
  \item Taktiken
  \item Puzzle lösen
  \item Eröffnungen
  \item Schach als Kunstform erkennen
  \item Schachvarianten
\end{itemize}

\pagebreak

\section{Regeln}%
\label{sec:regeln}

\subsection{Grundspielregeln}%
\label{sub:grundspielregeln}

\paragraph{Artikel 1: Wesen und Ziele des Schachspiels}

\begin{enumerate}[{1}.1]
  \item Das Schachspiel wird zwischen zwei Gegnern gespielt, die ihre Figuren auf einem
    quadratischen Spielbrett, \emph{Schachbrett} genannt, ziehen.
  \item Der Spieler mit den hellen Figuren (Weiß) führt den ersten Zug aus, dann ziehen
    die Spieler abwechselnd, wobei der Spieler mit den dunklen Figuren (Schwarz) den
    nächsten Zug ausführt.
    \marginnote{Dabei gilt auch Zugpflicht.}
  \item Ein Spieler ist am \emph{am Zug}, sobald der Zug seines Gegners ausgeführt worden
    ist.
  \item Das Ziel eines jeden Spielers ist es, den gegnerischen König so
    \emph{anzugreifen}, dass der Gegner keinen regelmäßigen Zug zur Verfügung hat.
    \marginnote{Also eben \emph{nicht} den gegnerischen König schlagen. Alternativ wäre
    möglich: den gegnerischen König so anzugreifen, dass dieser weder schlagen,
    blockieren oder ausweichen kann. Denn das sind die möglichen Züge um dem Schachgebot
    zu entkommen. Daher die Quizfrage: wie kann man dem Schachgebot entkommen?}
    \begin{enumerate}[{1.4}.1]
      \item Der Spieler, der dieses Ziel erreicht, hat den gegnerischen König \emph{matt
        gesetzt} und das Spiel gewonnen. Es ist nicht erlaubt, den eigenen König im Angriff
        stehen zu lassen, den eigenen König einem Angriff auszusetzen oder den König des
        Gegners zu schlagen.
      \item Der Gegner, dessen König matt gesetzt worden ist, hat das Spiel verloren.
    \end{enumerate}
  \item Ist eine Stellung erreicht, in der keinem der beiden Spieler das Mattsetzen des
    gegnerischen Königs mehr möglich ist, ist das Spiel \emph{remis}
    (unentschieden siehe Artikel 5.2.2).
\end{enumerate}

\paragraph{Artikel 2: Die Anfangsstellung der Figuren auf dem Schachbrett}%

\begin{enumerate}[{2}.1]
  \item Das Schachbrett besteht aus einem $8 \times 8$ Gitter von $64$ gleich großen
    Quadraten, die abwechselnd hell und dunkel sind (die \emph{weißen} und die
    \emph{schwarzen} Felder).
    Das Schachbrett wird so zwischen die beiden Spieler gelegt, dass auf der Seite vor
    einem Spieler das rechte Eckfeld weiß ist.
    \marginnote{Merke: alphabetisch kommt erst \emph{schwarz} und dann \emph{weiß}, daher
    ist die linke Ecke schwarz und die rechte weiß. Diese Eselsbrücke funktioniert auch
    auf Englisch.}

  \item Zu Beginn der Partie hat der eine Spieler 16 helle (\emph{weiße}), der andere 16
    dunkle (\emph{schwarze}) Figuren.

    Diese Figuren sind die folgenden:
    \marginnote{Auf Englisch: King, Queen, Rook, Bishop, Knight, Pawn.}
    \begin{center}
      \begin{tabular}{llll}
        \toprule
        ein weißer König    & \WhiteKingOnWhite   & K & K\\
        eine weiße Dame     & \WhiteQueenOnWhite  & D & Q\\
        zwei weiße Türme    & \WhiteRookOnWhite   & T & R\\
        zwei weiße Läufer   & \WhiteBishopOnWhite & L & B\\
        zwei weiße Springer & \WhiteKnightOnWhite & S & N\\
        acht weiße Bauern   & \WhitePawnOnWhite          \\
        \midrule
        ein schwarzer König    & \BlackKingOnWhite    & K & K\\
        eine schwarze Dame     & \BlackQueenOnWhite   & D & Q\\
        zwei schwarze Türme    & \BlackRookOnWhite    & T & R\\
        zwei schwarze Läufer   & \BlackBishopOnWhite  & L & B\\
        zwei schwarze Springer & \BlackKnightOnWhite  & S & N\\
        acht schwarze Bauern   & \BlackPawnOnWhite           \\
        \bottomrule
      \end{tabular}
    \end{center}

  \item Die Anfangsstellung der Figuren auf dem Schachbrett ist die folgende:
    \marginnote{Merke: die Dame steht auf dem Feld ihrer Farbe.}
    \begin{center}
      \newchessgame[]
      \chessboard[showmover=false]
    \end{center}

  \item Die acht senkrechten Spalten von Feldern heißen \emph{Linien}; die acht
    waagerechten Zeilen von Feldern heißen \emph{Reihen}. Eine geradlinige Folge von
    Felder gleicher Farbe, von einem Rand des Schachbrettes zum benachbarten Rand
    verlaufend, heißt \emph{Diagonale}.
\end{enumerate}

\paragraph{Artikel 3: Die Gangart der Figuren}%

\begin{enumerate}[{3}.1]
  \item Es ist nicht gestattet eine Figur auf ein Feld zu ziehen, das bereits von einer
    Figur der gleichen Farbe besetzt ist.

    \begin{enumerate}[{3.1}.1]
      \item Wenn eine Figur auf ein Feld zieht, das von einer gegnerischen Figur besetzt
        ist, wird letztere geschlagen und als Teil desselben Zuges vom Schachbrett
        entfernt.
      \item Eine Figur greift eine gegnerische Figur an, wenn sie auf jenem Feld gemäß
        Artikel 3.2 bis 3.8 schlagen könnte.
      \item Eine Figur greift ein Feld an, auch wenn sie am Zug gehindert ist, weil sie
        andernfalls den eigenen König im Angriff stehen lassen oder ihn einem Angriff
        aussetzen würde.
        \begin{marginfigure}
          \begin{center}
            \newchessgame[
              setfen=6Q1/1k4bR/2pP4/2B5/2B5/8/Rp6/K7 w - - 0 1 % chktex 8
              moveid=1w
            ]
            \chessboard[
              style=standard,
              smallboard,
            ]
          \end{center}

          Weiß steht im Schach. Wenn der weiße Turm den Bauern schlägt, dann ist dieser
          durch den Läufer gefesselt, setzt aber trotzdem den schwarzen König matt.
        \end{marginfigure}
    \end{enumerate}

  \item\label{regeln:laeufer}
    Der Läufer darf auf ein beliebiges anderes Feld entlang einer der Diagonalen
    ziehen, auf der er steht.
    \begin{center}
      \chessboard[
        setpieces={be4},
        showmover=false,
        padding=-0.8ex,
        pgfstyle={[fill]circle},
        markfields={d5, c6, b7, a8, f3, g2, h1, b1, c2, d3, f5, g6, h7}
      ]
    \end{center}

  \item Der Turm darf auf ein beliebiges Feld entlang der Linie oder der Reihe ziehen, auf
    der er steht.
    \begin{center}
      \chessboard[
        setpieces={rd3},
        showmover=false,
        padding=-0.8ex,
        pgfstyle={[fill]circle},
        markfields={d1, d2, d4, d5, d6, d7, d8, a3, b3, c3, e3, f3, g3, h3}
      ]
    \end{center}

  \item Die Dame darf auf ein beliebiges anderes Feld entlang der Linie, der Reihe oder
    einer der Diagonalen ziehen, auf der sie steht.
    \begin{center}
      \chessboard[
        setpieces={qe4},
        showmover=false,
        padding=-0.8ex,
        pgfstyle={[fill]circle},
        markfields={
          a4, b4, c4, d4, f4, g4, h4,
          e1, e2, e3, e5, e6, e7, e8,
          a8, b7, c6, d5, f3, g2, h1,
          b1, c2, d3, f5, g6, h7
        }
      ]
    \end{center}

  \item Beim Ausführen dieser Züge dürfen Dame, Turm und Läufer nicht über dazwischen
    stehende Figuren hinweg ziehen.

  \item Der Springer darf auf eines der Felder ziehen, die seinem Standfeld am nächsten,
    aber nicht auf gleicher Linie, Reihe oder Diagonalen mit diesem liegen.
    \begin{marginfigure}[10ex]
      \begin{center}
        \chessboard[
          smallboard,
          setpieces={Nd4},
          showmover=false,
          padding=-0.8ex,
          markstyle=circle,
          markfields={b3, b5, c2, c6, e2, e6, f3, f5},
          addpgf={
            \tikz[overlay] \draw [Maroon, line width=0.1em] (d4) circle (2.236em);
          },
        ]

      Der Springer springt im Kreis.
      \end{center}
    \end{marginfigure}
    \begin{center}
      \chessboard[
        setpieces={Nc3, bf8, ng8, rh8, pf7, pg7, ph7},
        showmover=false,
        padding=-0.8ex,
        markstyle=circle,
        markfields={a2, b1, a4, b5, d5, e4, e2, d1},
        pgfstyle={[fill]circle},
        markfields={e7, f6, h6},
      ]
    \end{center}

  \item
    Der Bauer
  \begin{enumerate}[{3.7}.1]
    \item darf vorwärts auf das Feld direkt vor ihm auf derselben Linie ziehen,
      sofern dieses Feld nicht besetzt ist, oder
    \item in seinem ersten Zug entweder wie unter 3.7.1 ziehen oder um zwei
      Felder auf derselben Linie vorrücken, sofern beide Felder nicht besetzt sind, oder
    \item auf ein von einer gegnerischen Figur besetztes Feld diagonal vor ihm auf einer
      benachbarten Linie ziehen, indem er die Figur schlägt.

      \begin{center}
        \chessboard[
          setpieces={Pc2, pg5},
          showmover=false,
          padding=-0.8ex,
          markstyle=circle,
          markfields={c3, c4},
          pgfstyle={[fill]circle},
          markfields={g4},
          markstyle=cross,
          shorten=0.6ex,
          markfields={b3, d3, f4, h4},
        ]
      \end{center}

    \begin{enumerate}[{3.7.4}.1]
      \item Ein Bauer, der auf derselben Reihe auf einem unmittelbar angrenzenden Feld wie
        ein gegnerischer Bauer, der soeben zwei Felder von seiner Anfangsstellung
        vorgerückt ist, steht, darf diesen gegnerischen Bauern so schlagen, als ob
        letzterer nur um ein Feld vorgerückt wäre.

      \item Dieses Schlagen ist nur in dem unmittelbar nachfolgenden Zug regelmäßig und
        wird \emph{Schlagen en passant} genannt.

      \begin{center}
        \chessboard[
          setpieces={Pd5, pe7},
          showmover=false,
          padding=-0.8ex,
          pgfstyle={[fill]circle},
          markfields={e5},
          markstyle=cross,
          shorten=0.6ex,
          markfields={e6},
        ]
      \end{center}
    \end{enumerate}

    \begin{enumerate}[{3.7.5}.1]
      \item Wenn ein Spieler, der am Zug ist, seinen Bauern auf die von seiner
        Anfangsstellung entfernteste Reihe zieht, muss er diesen als Teiler desselben
        Zuges gegen eine Dame, einen Turm, Läufer oder Springer derselben Farbe auf dem
        Ankunftsfeld austauschen. Dieses wird Umwandlungsfeld genannt.
      \item Die Auswahl des Spielers ist nicht auf bereits geschlagene Figuren beschränkt.
      \item Dieser Austausch eines Bauern für eine andere Figur wird \emph{Umwandlung}
        genannt. Die Wirkung der neuen Figur tritt sofort ein.
    \end{enumerate}

  \end{enumerate}

  \item Es gibt zwei verschiedene Arten den König zu ziehen:
    \begin{enumerate}[{3.8}.1]
      \item durch Ziehen auf ein beliebiges angrenzendes Feld,

      \begin{center}
        \chessboard[
          setpieces={Kc3, ke8},
          showmover=false,
          padding=-0.8ex,
          markstyle=circle,
          markfields={b2, b3, b4, c2, c4, d2, d3, d4},
          pgfstyle={[fill]circle},
          markfields={d7, d8, e7, f7, f8},
        ]
      \end{center}

      \item durch Rochieren.
        Die Rochade ist ein Zug des Königs und eines gleichfarbigen
        Turmes auf der ersten Reihe des Spielers, der als ein Königszug gilt und
        folgendermaßen ausgeführt wird: Der König wird von seiner Anfangsstellung
        um zwei Felder in Richtung des Turmes, der auf seiner Anfangsstellung stehen muss,
        hin versetzt; dann wird dieser Turm auf das Feld gesetzt, das der König soeben
        überquert hat.
        \begin{marginfigure}
          \begin{center}
            \newchessgame[
              setfen=3k/8/8/8/8/8/1r/R3K w Q - 0 1 % chktex 8
              moveid=1w
            ]
            \chessboard[
              style=standard,
              smallboard,
            ]
          \end{center}
          Bester Zug für Weiß?
        \end{marginfigure}


      \begin{center}
        \begin{tabular}{ll}
          \chessboard[
            fontsize=13pt,
            labelleft=false,
            labelbottom=false,
            margin=false,
            setpieces={Ke1, Rh1, ke8, ra8},
            showmover=false,
            padding=-0.8ex,
          ]

          &

          \chessboard[
            fontsize=13pt,
            labelleft=false,
            labelbottom=false,
            margin=false,
            setpieces={Kg1, Rf1, kc8, rd8},
            showmover=false,
            padding=-0.8ex,
          ]
          \\
          Vor weißer kurzer Rochade & Nach weißer kurzer Rochade\\
          Vor schwarzer langer Rochade & Nach schwarzer langer Rochade\\[2ex]

          \chessboard[
            fontsize=13pt,
            labelleft=false,
            labelbottom=false,
            margin=false,
            setpieces={Ke1, Ra1, ke8, rh8},
            showmover=false,
            padding=-0.8ex,
          ]

          &

          \chessboard[
            fontsize=13pt,
            labelleft=false,
            labelbottom=false,
            margin=false,
            setpieces={Kc1, Rd1, kg8, rf8},
            showmover=false,
            padding=-0.8ex,
          ]
          \\
          Vor weißer langer Rochade & Nach weißer langer Rochade\\
          Vor schwarzer kurzer Rochade & Nach schwarzer kurzer Rochade\\
        \end{tabular}
      \end{center}

      \begin{enumerate}[{3.8.2}.1]
        \item Das Recht zu rochieren ist verloren:
          \begin{enumerate}[{3.8.2.1}.1]
            \item wenn der König bereits gezogen hat, oder
            \item mit einem Turm, der bereits gezogen hat.
          \end{enumerate}
      \end{enumerate}

      \begin{enumerate}[{3.8.2}.2]
        \item Die Rochade ist vorübergehend verhindert,
          \begin{enumerate}[{3.8.2.2}.1]
            \item wenn das Standfeld des Königs oder das Feld, das er überqueren muss,
              oder sein Zielfeld von einer oder mehreren gegnerischen Figuren angegriffen
              wird,
            \item wenn sich zwischen dem König und dem Turm, mit dem rochiert werden soll,
              irgendeine Figur befindet.
          \end{enumerate}
      \end{enumerate}
    \end{enumerate}

  \begin{enumerate}[{3.9}.1]
    \item Ein König steht \emph{im Schach}, wenn er von einer oder mehreren gegnerischen
      Figuren angegriffen wird, auch wenn diese selbst nicht auf das vom König besetzte Feld
      ziehen können, weil sie anderenfalls den eigenen König im Angriff stehen lassen oder
      diesen einem Angriff aussetzen würden.
    \item Keine Figur darf einen Zug machen, der entweder den König derselben Farbe einem
      Schachgebot aussetzt oder diesen in einem Schachgebot stehen lässt.
  \end{enumerate}

  \begin{enumerate}[{3.10}.1]
    \item Ein Zug ist regelmäßig, wenn die maßgeblichen Bedingungen der Artikel 3.1 bis
      3.9 erfüllt wurden.
    \item Ein Zug ist regelwidrig, wenn er eine der maßgeblichen Bedingungen der Artikel
      3.1 bis 3.9 nicht erfüllt.
    \item Eine Stellung ist regelwidrig, wenn sie nicht durch irgendeine Folge
      regelgemäßer Züge erreicht werden kann.
  \end{enumerate}
\end{enumerate}

\paragraph{Weitere ausgewählte Regeln}%
\label{par:weitere_ausgewahlte_regeln}

\paragraph{4.1} Jeder Zug muss mit einer Hand alleine ausgeführt werden.
\paragraph{4.2.1} Nur der Spieler, der am Zug ist, darf eine oder mehrere Figuren auf
ihren Felder zurechtrücken, vorausgesetzt, dass er seine Absicht im Voraus bekannt gibt
(zum Beispiel durch die Ankündigung \emph{j’adoube} oder \emph{ich korrigiere}).
\paragraph{4.2.2} Jede andere Berührung einer Figur gilt als absichtliche Berührung, außer
dies geschieht offensichtlich aus Versehen.
\paragraph{4.4.2} Wenn der am Zug befindliche Spieler absichtlich seinen Turm und danach
seinen König berührt, darf er mit diesem Turm in diesem Zug nicht rochieren und der Fall
wird durch Artikel 4.3.1 geregelt.
\paragraph{5.1.1} Die Partie ist von dem Spieler gewonnen, der den gegnerischen König
mattgesetzt hat. Damit ist die Partie sofort beendet, vorausgesetzt, dass der Zug, der die
Mattstellung herbeigeführt hat, mit Artikel 3 und den Artikeln 4.2 bis 4.7 übereinstimmte.
\paragraph{5.1.2} Die Partie ist von dem Spieler gewonnen, dessen Gegner erklärt, dass er
aufgebe. Damit ist die Partie sofort beendet.
\paragraph{5.2.1} Die Partie ist remis, wenn der Spieler, der am Zug ist, keinen
regelgemäßen Zug zur Verfügung hat und sein König nicht im Schach steht. Eine solche
Stellung heißt \emph{Pattstellung}. Damit ist die Partie sofort beendet, vorausgesetzt,
dass der Zug, der die Mattstellung herbeigeführt hat, mit Artikel 3 und den Artikeln 4.2
bis 4.7 übereinstimmte.
\paragraph{5.2.2} Die Partie ist remis, sobald eine Stellung entstanden ist, in welcher
keiner der Spieler den gegnerischen König mit irgendeiner Folge regelmäßiger Züge matt
setzen kann. Eine solche Stellung heißt \emph{tote Stellung}.
\marginnote{Mögliche tote Stellungen: KL–K, KS–K, KSS–K}
Damit ist die Partie sofort beendet, vorausgesetzt, dass der Zug, der die Mattstellung
herbeigeführt hat, mit Artikel 3 und den Artikeln 4.2 bis 4.7 übereinstimmte.
\paragraph{5.2.3} Die Partie ist remis durch eine von den beiden Spielern während der
Partie getroffene Übereinkunft, sofern beide Spieler mindestens einen Zug ausgeführt
haben. Damit ist die Partie sofort beendet.
\paragraph{9.2.1} Die Partie ist remis aufgrund eines korrekten Antrages des Spielers, der
am Zug ist, wenn die gleiche Stellung mindestens zum dritten Mal (nicht notwendigerweise
durch Zugwiederholung)
\paragraph{9.2.1.1} sogleich entstehen wird, falls der Spieler als erstes seinen Zug, der
nicht geändert werden kann, auf sein Partieformular schreibt und dem Schiedsrichter seine
Absicht erklärt, diesen Zug ausführen zu wollen, oder
\paragraph{9.2.1.2} soeben entstanden ist und der Antragsteller am Zug ist.
\paragraph{9.2.2} Stellungen gelten nur dann als gleich, wenn derselbe Spieler am Zug ist,
Figuren der gleichen Art und Farbe die gleichen Felder besetzen und die Zugmöglichkeiten
aller Figuren beider Spieler gleich sind. Demgemäß sind Stellungen nicht gleich, wenn
\paragraph{9.2.2.1} ein Bauer zu Beginn der Zugfolge \emph{en passant} geschlagen werden
konnte, oder
\paragraph{9.2.2.2} ein König das Recht zur Rochade mit einem Turm, der noch nicht bewegt
worden ist, hatte, dieses aber nach dem Zug verloren hat. Das Rochaderecht geht erst
verloren, nachdem der König oder Turm gezogen hat.
\paragraph{9.3} Die Partie ist remis aufgrund eines korrekten Antrages des Spieler, der am
Zug ist, falls
\paragraph{9.3.1} er einen Zug, der nicht geändert werden kann, auf sein Partieformular
schreibt und dem Schiedsrichter seine Absicht erklärt, diesen Zug ausführen zu wollen, der
zur Folge habe, dass dann die letzten 50 aufeinanderfolgenden Züge eines jeden Spieler
ausgeführt worden sind, ohne dass ein Bauer gezogen hat und ohne dass eine Figur
geschlagen worden ist, oder
\paragraph{9.3.2} die letzten 50 aufeinanderfolgenden Züge von jedem Spieler abgeschlossen
worden sind, ohne dass ein Bauer gezogen hat und ohne dass eine Figure geschlagen worden
ist.
\paragraph{9.6} Falls eine oder beide der folgenden Situationen auftreten, ist die Partie
remis:
\paragraph{9.6.1} sobald eine gleiche Stellung, entsprechend Artikel 9.2.2, mindestens
fünfmal entstanden ist,
\paragraph{9.6.2} sobald wenigstens 75 Züge von jedem Spieler ausgeführt worden sind, ohne
dass ein Bauer gezogen hat und ohne dass eine Figur geschlagen worden ist.
Wenn der letzte Zug matt setzt, hat dies Vorrang.
\paragraph{11.1} Die Spieler dürfen nichts unternehmen, das dem Ansehen des Schachspiels
abträglich sein könnte.
\paragraph{11.5} Es ist verboten, den Gegner auf irgendwelche Art abzulenken oder zu
stören. Dazu gehört auch ungerechtfertigtes Antragstellen oder ungerechtfertigtes
Anbieten von Remis oder das Mitbringen einer Geräuschquelle in den Spielbereich.
\paragraph{11.7} Andauernde Weigerung eines Spieler, sich an die Schachregeln zu halten,
wird mit Partieverlust bestraft. Die vom Gegner erzielte Punktzahl wird vom Schiedsrichter
bestimmt.
\paragraph{11.8} Wenn sich beide Spieler gemäß Artikel 11.7 schuldig machen, wird für
beide das Spiel für verloren erklärt.

\pagebreak

\section{Matt setzen}%
\label{sec:matt_setzen}

(siehe \texttt{mate\_puzzles.pdf})

\section{Endspiele}%
\label{sec:endspiele}

\subsection{Elementare Mattführungen}%
\label{sub:elementare_mattfuhrungen}

\begin{itemize}
  \item mit Dame und Turm
  \item mit zwei Türmen
  \item mit einer Dame
  \item mit einem Turm
  \item mit zwei Läufern
  \item mit einer Dame gegen einen Läufer
  \item mit einer Dame gegen einen Springer
  \item mit einem Bauern
  \item nur wer richtig gut ist: mit Läufer und Springer
\end{itemize}

\subsection{Bauernendspiel}%
\label{sub:bauernendspiel}

\paragraph{Quadratregel}%
\label{par:quadratregel}

Betrachte die Endspielstellung am Rand. Kann Schwarz den Bauern noch einholen?

\begin{marginfigure}
  \begin{center}
    \newchessgame[
      setfen=8/8/8/8/5P/k/8/K w - - 0 1 % chktex 8
      moveid=1w
    ]
    \chessboard[
      smallboard,
      showmover=false,
    ]
  \end{center}
\end{marginfigure}

Es hängt davon ab, wer am Zug ist. Wenn Schwarz am Zug ist, dann kann der König das
\emph{Quadrat} noch betreten und holt den Bauern ein. Wenn Weiß am Zug ist, dann nicht,
denn dann ist der Bauer eher am Umwandlungsfeld. Das \emph{Quadrat} wird vom Bauernfeld
bis zum Umwandlungsfeld gezogen. Es gilt also: wer am Zug ist gewinnt.

\begin{center}
  \newchessgame[
    setfen=8/8/8/8/5P/k/8/K w - - 0 1 % chktex 8
    moveid=1w
  ]
  \chessboard[
    smallboard,
    shorten=0.6ex,
    showmover=false,
    markstyle=cross,
    markfields={f5, f6, f7, e4, d4, c4, b5, b6, b7, b8, c8, d8, e8},
    padding=-0.8ex,
    pgfstyle={[fill]circle},
    markfields={b4, c5, d6, e7, f8},
  ]
\end{center}

\paragraph{Nur ein Bauer}%
\label{par:nur_ein_bauer}

Zauberwort: \emph{Opposition}. Betrachten wir zuerst die \emph{direkte Opposition}. Dabei
stehen sich die König auf einer Linie gegenüber, sodass nur ein Feld zwischen ihnen ist.
Wer \emph{nicht} am Zug ist, hat die Opposition.

\begin{center}
  \newchessgame[
    setfen=8/8/8/3k/8/3K/8/8 w - - 0 1 % chktex 8
    moveid=1w
  ]
  \chessboard[
    smallboard,
    shorten=0.6ex,
    showmover=false,
    markstyle=cross,
    markfields={c4, d4, e4},
  ]
\end{center}

Dabei kämpfen die Könige um die Kontrolle der markierten Felder. Wer am Zug ist, muss eins
der Felder aufgeben. Daraufhin kann der andere König flankieren.

\begin{center}
  \chessboard[
    style=standard,
    smallboard,
    markmoves={d5-c5, d3-e4},
    showmover=false,
  ]
\end{center}

Betrachte folgende Ausgangsposition für dieses Endspiel:
\begin{center}
  \newchessgame[
    setfen=8/8/8/3k/8/8/3PK/8 w - - 0 1 % chktex 8
    moveid=1w
  ]
  \chessboard[
    smallboard,
    showmover=false,
  ]
\end{center}
Wer am Zug ist, gewinnt durch Nehmen der Opposition.
\begin{center}
  \chessboard[
    style=standard,
    smallboard,
    markmoves={d5-e4, e2-d3},
    showmover=false,
  ]
\end{center}

\paragraph{Bauerndurchbruch}%
\label{par:bauerndurchbruch}

Siehe \href{https://de.wikipedia.org/wiki/Elementare_Mattführung}{WP: elementare
Mattführung}

\subsection{Bring it home}%
\label{sub:bring_it_home}

\begin{itemize}
  \item \url{https://lichess.org/rqVQFoMn/black#85}
\end{itemize}

\subsection{Lucena-Stellung}%
\label{sub:lucena_stellung}

Siehe \url{https://lichess.org/study/Vnzt6p8p}

\subsection{Philidors Stellung}%
\label{sub:philidors_stellung}

\section{Taktik}%
\label{sec:taktik}

\begin{notiz}
  \begin{itemize}
    \item ungedeckte Figuren
    \item Gabel
    \item Zwischenzug, Zwischenschach
    \item Fesselung
    \item Überlastung
    \item Opfer
    \item Fangen von Figuren
  \end{itemize}
\end{notiz}

\section{Schachprobleme}%
\label{sec:schachprobleme}

\paragraph{Paul Morphy, 1856}%

\begin{center}
  \newchessgame[
    setfen=kbK/pp/1P/8/8/8/8/R w - - 0 1 % chktex 8
    moveid=1w
  ]
  \chessboard[
    style=standard,
  ]
\end{center}

\paragraph{Leonid Iwanowitsch Kubbel, 1924}%

\begin{center}
  \newchessgame[
    setfen=N5Q/nkPP/8/8/4K/8/8/8 w - - 0 1 % chktex 8
    moveid=1w
  ]
  \chessboard[
    style=standard,
  ]
\end{center}

\section{Berühmte Partien}%
\label{sec:beruhmte_partien}

\paragraph{Napoleon – der Schachautomat}%

\newchessgame

\mainline{1. e4 e5 2. Qf3 Nc6 3. Bc4 Nf6 4. Ne2 Bc5 5. a3 d6 6. O-O Bg4 7. Qd3 Nh5 8. h3
Bxe2 9. Qxe2 Nf4 10. Qe1 Nd4 11. Bb3 Nxh3+ 12. Kh2 Qh4 13. g3 Nf3+ 14. Kg2
Nxe1+ 15. Rxe1 Qg4 16. d3 Bxf2 17. Rh1 Qxg3+ 18. Kf1 Bd4 19. Ke2 Qg2+ 20. Kd1
Qxh1+ 21. Kd2 Qg2+ 22. Kd1 Ng1 23. Nc3 Bxc3 24. bxc3 Qe2#}

\paragraph{Morphy – Karl von Braunschweig und Graf Isoard, Paris 1858}%

\newchessgame

Auch bekannt als \textit{The Opera Game}, weil sie während der Oper \textit{Der Barbier
von Sevilla} gespielt wurde.

Es geht los mit \mainline{1. e4 e5 2. Nf3 d6}, die Philidor-Verteidigung.
Weiß attackiert mit \mainline{3. d4} das Zentrum.
Schwarz versucht mit \mainline{3... Bg4} den Springer an die Dame zu fesseln.
Das ist jedoch ein Fehler, weil Weiß mit \variation{4. dxe5 dxe5 5. Qxd8+ Kxe8 6. Nxe5}
den Bauern und das Rochaderecht verliert.
\begin{marginfigure}
  \chessboard[
    style=standard,
    boardfontsize=8pt,
    labelleft=false,
    labelbottom=false
  ]
\end{marginfigure}
Daher spielt Morphy auch \mainline{4. dxe5} und Schwarz antwortet mit \mainline{4...
Bxf3}. Nach \mainline{5. Qxf3 dxe5 6. Bc4} bedroht Weiß f7 mit Matt.
\begin{marginfigure}
  \chessboard[
    style=standard,
    boardfontsize=8pt,
    labelleft=false,
    labelbottom=false
  ]
\end{marginfigure}
Schwarz verteidigt mit \mainline{6... Nf6}, woraufhin Weiß mit \mainline{7. Qb3}
antwortet.
Damit wird weiterhin Druck auf f7 ausgeübt und sogar Matt in 2 angedroht.
Weiterhin ist der Bauer auf b7 ungedeckt und wird jetzt durch die Dame angegriffen.
Schwarz antwortet mit \mainline{7... Qe7} und möchte damit, falls Weiß sich mit
\variation{8. Qxb7} den Bauern holt, mittels \variation{8... Qb4+} einen Damentausch
erzwingen.
Weiß hat daran kein Interesse und will die Initiative behalten.
Dazu werden weitere Truppen ausgehoben: \mainline{8. Nc3}.
Damit wird auch das drohende Schachgebot von \variation{8... Qb4+} unterbunden.
Schwarz verteidigt den Bauern auf b7 durch die Dame mit \mainline{8... c6}.
Morphy entwickelt die letzte Leichtfigur mit \mainline{9. Bg5} und fesselt den Springer an
die Dame.
\begin{marginfigure}
  \chessboard[
    style=standard,
    boardfontsize=8pt,
    labelleft=false,
    labelbottom=false
  ]
\end{marginfigure}
Schwarz setzt zum Gegenangriff mit \mainline{9... b5} an.
Weiß kann sich durch die schnelle Entwicklung das folgende Opfer leisten:
\mainline{10.  Nxb5 cxb5 11. Bxb5+ Nbd7}, wodurch der schwarze Springer auch gefesselt ist.
Weiß bringt mit \mainline{12. O-O-O} die letzten Truppen ins Gefecht und Schwarz
verteidigt den gefesselten Springer mit \mainline{12... Rd8}. Dies engt den schwarzen
König jedoch sehr ein.
\begin{marginfigure}
  \chessboard[
    style=standard,
    boardfontsize=8pt,
    labelleft=false,
    labelbottom=false
  ]
\end{marginfigure}
Nach dem Springeropfer folgt jetzt das Turmopfer, Morphy haut alles rein, was er hat:
\mainline{13. Rxd7! Rxd7 14. Rd1 Qe6}.
Auf den letzten Meter bietet Schwarz erneut einen Damentausch an, aber dafür ist es zu
spät. Weiß setzt die Attacke mit \mainline{15. Bxd7 Nxd7} fort und beendet sie mit einem
wunderschönen Damenopfer: \mainline{16. Qb8+!! Nxb8 17. Rd8#}.

\section{Die Schachelemente}%
\label{sec:die_schachelemente}

\subsection{Material}%
\label{sub:material}

\subsection{Zeit}%
\label{sub:zeit}

Siehe \url{https://www.youtube.com/watch?v=c3B34oKM0oI}

\subsection{Raum}%
\label{sub:raum}

\begin{itemize}
  \item Wenn du nicht das Zentrum kontrollierst, wirst du eingeengt.
  \item Eine eingeengte Stellung lässt sich schwer spielen, da die Figuren sich
    gegenseitig blockieren.
  \item In einer eingeengten Stellung gibt es kaum Möglichkeiten für Konter.
  \item Falsch platzierte Figuren sind anfällig für Taktiken.
\end{itemize}

\subsection{Bauernstruktur}%
\label{sub:bauernstruktur}

\subsection{Sicherheit des Königs}%
\label{sub:sicherheit_des_konigs}

\section{Eröffnungen}%
\label{sec:eroffnungen}

\begin{notiz}
  Ziele der Eröffnung, Raum – Zeit – Material – Sicherheit des Königs, Wie erreichen
  bestimmte Eröffnungen diese Ziele?
\end{notiz}

\subsection{Wichtige Eröffnungen}%
\label{sub:wichtige_eroffnungen}

\begin{itemize}
  \item Italienische Partie

    \href{https://de.wikipedia.org/wiki/Steinitz_–_von_Bardeleben,_Hastings_1895}{Steinitz
    – von Bardeleben, Hastings 1895}
  \item Spanische Partie
  \item Damengambit
  \item Sizilianische Verteidigung
\end{itemize}

\subsection{Problematische Eröffnungen}%
\label{sub:problematische_eroffnungen}

\begin{itemize}
  \item \url{https://en.wikipedia.org/wiki/Greco_Defence}
\end{itemize}

\subsection{Eröffnungsfallen}%
\label{sub:eroffnungsfallen}

\paragraph{Das Seekadettenmatt}%
\label{par:das_seekadettenmatt}

\newchessgame

Im Englischen als \textit{Légal’s Trap} bekannt. Die Falle beginnt mit der italienischen
Eröffnung: \mainline{1. e4 e5 2. Nf3 Nc6 3. Bc4} und dann das „Semi-Italian Opening“ mit
\mainline{3... d6}.

Nach der Figurenentwicklung \mainline{4. Nc3} will Schwarz den Springer auf f3 mittels
\mainline{4... Bg4?!} an die Dame fesseln. Und jetzt kommt die Falle: \mainline{5. Nxe5?}.
Wenn Schwarz gierig ist, dann wird die Dame geschlagen: \mainline{5... Bxd1??}.
\begin{marginfigure}
  \begin{center}
    \chessboard[
      tinyboard,
    ]
  \end{center}
\end{marginfigure}
Das führt zu einem Matt in 2: \mainline{6. Bxf7+ Ke7} ist der einzige Zug und \mainline{7.
Nd5#} ist Matt.

\begin{center}
  \chessboard[smallboard]
\end{center}

Schwarz hätte nicht so gierig sein dürfen, sondern hätte \variation{5... Nxe5} spielen
sollen.

\pagebreak

\section{Schachvarianten}%
\label{sec:schachvarianten}

\begin{itemize}
  \item Fischer Random Chess bzw. Chess960
  \item Atomic Chess
  \item Crazyhouse
  \item Horde
  \item Antichess
  \item King of the Hill
  \item Racing Kings
\end{itemize}

\section{Planung \& Reflexion}%
\label{sec:planung_reflexion}

\begin{description}
  \item[17.10.2019] Regeln, Endspiele (KQR-K, KRR-K)

    Die Erklärung der Regeln hat zu lange gedauert.
    Die Übungen zwischendurch waren gut.
    Nächstes Mal die Regeln eindampfen.

  \item[24.10.2019] Nachtrag zur komischen Regel (Beispielposition zeigen),
    wiederholende Übungen zu Regeln,
    Matt in 1 Puzzles (Seite 1),
    neue Endspiele (KQ-K, KR-K, KBB-K),
    gegen den Computer die Endspiele üben,
    Lichess zeigen und Fernschach anleiern,
    freies Spielen vereinfachter Varianten

  \item[07.11.2019] Matt in 1 Puzzles (Seite 2),
    neues Endspiel (KLL-K),
    gegeneinander üben,
    gegen Computer zeigen,
    freies Spiel

  \item[14.11.2019] Matt in 1 Puzzles (Seite 3),
    Endspiel (KD-KL, KD-KS),
    gegeneinander üben,
    freies Spiel

  \item[21.11.2019] Matt in 1 Puzzles (Seite 4),
    Endspiel (KB-K),
    Opposition (direkt, diagonal, entfernte),
    gegeneinander üben,
    freies Spiel

  \item[28.11.2019] Matt in 1 Puzzles (Seite 5),
    Taktiktraining auf Lichess,
    in Paaren gemeinsam lösen und aufschreiben,
    freies Spiel

  \item[05.12.2019] Matt in 1 Puzzles (Seite 6),
    Kurzvortrag zur Algebraischen Notation,
    Taktiktraining auf Lichess,
    in Paaren gemeinsam lösen und aufschreiben,
    freies Spiel

  \item[12.12.2019] Matt in 1 Puzzles (Seite 7),
    Taktiktraining auf Lichess,
    in Paaren gemeinsam lösen und aufschreiben,
    freies Spiel
\end{description}

\end{document}
